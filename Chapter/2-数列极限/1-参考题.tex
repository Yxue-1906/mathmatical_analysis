
\section{参考题}
 \subsection{第一组}
     \begin{exercise}
         设$\{a_{2k-1}\},\{a_{2k}\},\{a_{3k}\}$都收敛,证明:$\{a_n\}$收敛.
     \end{exercise}
     \begin{solution}
         设$\{a_{2k-1}\}$,收敛到$p$,$\{a_{2k}\}$收敛到$q$,而$\{a_{3k}\}$中可以选出全属于$\{a_{2k-1}\}$或$\{a_{2k}\}$的子列(事实上$\{a_{3k}\}$中的项交替地从$\{a_{2k-1}\}$和$\{a_{2k}\}$中取出),由于收敛数列子列收敛于同一极限,也即$p=q$,而前证得$\{a_{2k-1}\},\{a_{2k}\}$收敛于同一极限则$\{a_n\}$也收敛于同一极限知$\{a_n\}$收敛.
     \end{solution}

     \begin{exercise}
         设$\{a_n\}$有界,且满足条件$a_n\leqslant a_{n+2},a_n\leqslant a_{n+3},n\in \mathbb{N}_+$,证明:$\{a_n\}$收敛.
     \end{exercise}
     \begin{solution}
         由题知$\{a_{2k-1}\},\{a_{2k}\},\{a_{3k}\}$收敛,由上一小题知命题成立.
     \end{solution}

     \begin{exercise}
         设$\{a_n+a_{n+1}\}$和$\{a_n+a_{n+2}\}$都收敛,证明:$\{a_n\}$收敛.
     \end{exercise}
     \begin{solution}
         由题知$\{a_{n+2}-a_{n+1}\}$也收敛,而实质上这与$\{a_{n+1}-a_n\}$没有区别,与$\{a_{n+1}+a_n\}$相减即得$\{a_n\}$收敛.
     \end{solution}

     \begin{exercise}
         设数列$\{a_n\}$收敛于0,又存在极限$\lim _{n\to\infty} \left| \frac {a_{n+1}}{a_n} \right|=a$.证明:$a\leqslant 1$.
     \end{exercise}
     \begin{solution}
         反证.

         首先$a_n\neq0$,否则$\frac{a_{n+1}}{a_n}$某项没有意义.

         若$a>1$,则$\exists\, a'$使得$1<a'<a$,而取$\varepsilon<a-a'$,无论多大的$N$,总$\exists\, n>N,a_n\geqslant a_N\cdot a^{n-N}>\varepsilon$,与$\{a_n\}$收敛到0矛盾.
     \end{solution}

     \begin{exercise}
         设$a_n=\sum\limits _{k=1}^n \left(\sqrt{1+\frac{k}{n^2}}-1\right)$,$n\in \mathbb{N}_+$,计算$\lim\limits _{n\to \infty}a_n$.
     \end{exercise}
     \begin{solution}
         放缩.

         $a_n=\sum_{k=1}^n \frac{\frac{k}{n^2}}{\sqrt{1+\frac{k}{n^2}}+1}$

         而有$\frac{n+1}{2n}\cdot \frac{1}{\sqrt{1+\frac{1}{n^2}}+1}\leqslant a_n \leqslant \frac{n+1}{2n}\cdot\frac{1}{\sqrt{1+\frac{1}{n}}+1}$(放成同一分母)

         令$n$趋向$\infty$由夹逼定理即知$\lim_{n\to\infty}a_n=\frac{1}{2}$.

         说明:直接放缩会放过$\Longrightarrow n\cdot\left(\sqrt{1+\frac{1}{n^2}}-1\right)=\frac{n\cdot\frac{1}{n^2}}{\sqrt{1+\frac{1}{n^2}}+1}\rightarrow 0$
     \end{solution}

     \begin{exercise}
         用$p(n)$表示能整除$n$的素数的个数,证明:$\lim _{n\to \infty}\frac{p(n)}{n}=0$.
     \end{exercise}
     \begin{solution}
         设$p_n$为第$n$个质数.可以这样估计一个用于夹逼的上数列:

         当$\prod_{i=1}^{k-1}p_i\leqslant n\leqslant\prod_{i=1}^kp_i$
         (也就是夹在$p_1p_2\cdots p_{k-1}$和$p_1p_2\cdots p_k$之间),

         则$\frac{p(n)}{n}\leqslant \frac{k}{p_1p_2\cdots p_k}<\frac{k}{k!}$,由夹逼定理知原极限为0.

         说明:事实上这就是一个对$p(n)$的一个估计,换个角度考虑,就是怎么使$\frac{p(n)}{n}$尽可能大,
         也就是变相估计上界吧.
     \end{solution}

     \begin{exercise}
         设$a_0,a_1,\cdots,a_p$是$p+1$个给定的数,且满足条件$a_0+a_1+\cdots+a_p=0$.求

         $\lim\limits _{n\to \infty}\left(a_0\sqrt{n}+a_1\sqrt{n+1}+\cdots+a_p\sqrt{n+p}\right)$.

     \end{exercise}
     \begin{solution}

         $\left|a_0\sqrt{n}+\cdots+a_p\sqrt{n+p}-0\right|=$

         $\left|a_0(\sqrt{n}-\sqrt{n})+a_1(\sqrt{n+1}-\sqrt{n})+\cdots+a_p(\sqrt{n+p}-\sqrt{n})\right|\leqslant$

         $\left|a_0(\sqrt{n}-\sqrt{n})\right|+\left|a_1(\sqrt{n+1}-\sqrt{n})\right|+\cdots=$

         $\left|a_0\cdot0\right|+\left|a_1\cdot\frac{1}{\sqrt{n+1}+\sqrt{n}}\right|+\cdots+\left|\frac{a_p}{\sqrt{n+p}+\sqrt{n}}\right|\to 0$

         故极限为0.
     \end{solution}

     \begin{exercise}
         证明:当$0<k<1$时,$\lim _{n\to\infty} \left[(1+n)^k-n^k\right]=0$.
     \end{exercise}
     \begin{solution}
         显然$0\leqslant(1+n)^k-n^k=n^k\left[(1+\frac{1}{n})^k-1\right]\leqslant n^k\left[(1+\frac{1}{n})-1\right]=\frac{n^k}{n}\to0$.
     \end{solution}
     \begin{note}
         另一思路也可以是证明单调有界,但是没想到怎么证明单调减...
     \end{note}

     \begin{exercise}

     \end{exercise}
     \begin{enumerate}
         \item 设$\{a_n\}$收敛,令$y_n=n(x_n-x_{n-1}),n\in \mathbb{N}_+$,问$\{y_n\}$是否收敛?
               \begin{solution}
                   不一定.
                   \[
                       y_n=
                       \begin{cases}
                           x_{n-1}+\frac{1}{n}, \, n\text{为完全平方数} \\
                           x_{n-1}, \, \text{其他}
                       \end{cases}
                   \]
                   可知$\{x_n\}$不超过$\sum_{n = 1}^{\infty}\frac{1}{n^2} $,前证它收敛.而对于$\varepsilon<1$,无论多大的$N$总$\exists\, n>N$且$y_n=1$,即$\{y_n\}$发散.
               \end{solution}

         \item 在上一小题中,若$\{y_n\}$也收敛,证明:$\{y_n\}$收敛于0.
               \begin{solution}
                   设$\{y_n\}$收敛到$a$,由Stolz知$\left\{\frac{x_n-x_{n-1}}{\frac{1}{n}}\right\}$极限若存在则与$\frac{x_n}{\sum_{i=1}^\infty\frac{1}{i}}$相同,显然此数列收敛到0(分母$\to\infty$),而由题知$\left\{\frac{x_n-x_{n-1}}{\frac{1}{n}}\right\}$收敛,也即$a=0$.
               \end{solution}
     \end{enumerate}

     \begin{exercise}
         \begin{enumerate}
             \item 设正数列$\{a_n\}$满足条件$\lim_{n\to\infty}\frac{a_n}{a_{n+1}}=0$,证明:$\{a_n\}$是正无穷大量.
                   \begin{solution}
                       事实上,当$n$足够大时$\left|\frac{a_n}{a_{n+1}}\right|=\frac{a_n}{a_{n+1}}<\varepsilon<1$.

                       则当$n>N$时,$\frac{1}{a_n}<\frac{1}{a_N}\cdot\varepsilon^{n-N}$,即$a_n>\frac{a_N}{\varepsilon^{n-N}}>M$成立,($M$为一给定的任意大的数),也即$\{a_n\}$为无穷大量.
                   \end{solution}

             \item 设正数列$\{a_n\}$满足条件$\lim_{n\to\infty}\frac{a_n}{a_{n+1}+a_{n+2}}=0$,证明:$\{a_n\}$无界.
                   \begin{solution}
                       反证.假设$\{a_n\}$有上界$M$;而因为$\lim_{n\to\infty}\frac{a_n}{a_{n+1}+a_{n+2}}=0$$\Rightarrow $对$\varepsilon=\frac{1}{4},\exists\, N$使得当$n>N$时$\frac{a_n}{a_{n+1}+a_{n+2}}<\varepsilon=\frac{1}{4}$.那么可以推出$4a_N<a_{N+1}+a_{N+2},16a_N<a_{N+2}+2a_{N+3}+a_{N+4},\cdots$.

                       因为$M$固定,故$\exists\, m$使得$2^ma_N>M$,即$2^mM<4^ma_N<\underbrace{\cdots}_{2^m\text{个}\{a_n\}\text{中的项}} $.

                       由抽屉原理知,至少有一个项$>M$,与假设矛盾.故原命题成立.

                       说明:只说无界是正确的,不一定是正无穷大量.$\{a_n\}$可以是$1!,1,2!,1,3!,1,\cdots$,这也满足题设.
                   \end{solution}
         \end{enumerate}
     \end{exercise}

     \begin{exercise}
         证明:$\left(\frac{n}{3}\right)^n<n!<\left(\frac{n}{2}\right)^n$,其中右边的不等式当$n\geqslant 6$时成立.
     \end{exercise}
     \begin{solution}
         与2.5练习题7.证法完全类似.
     \end{solution}

     \begin{exercise}
         证明:$\left(\frac{n}{e}\right)^n<n!<e\left(\frac{n}{2}\right)^n$.
     \end{exercise}
     \begin{solution}
         与2.5练习题7.证法完全类似.
     \end{solution}

     \begin{exercise}
         (对于命题2.5.4的改进)证明:
         \begin{enumerate}
             \item $n\geqslant2$时成立:$1+1+\frac{1}{2!}+\cdots+\frac{1}{n!}+\frac{1}{n!n}=3-\frac{1}{2!1\cdot2}-\cdots-\frac{1}{n!(n-1)n}$;
                   \begin{solution}
                       也即证$\frac{1}{2!}+\cdots+\frac{1}{n!}+\frac{1}{n!n}=1-\frac{1}{2!1\cdot2}-\cdots-\frac{1}{n!(n-1)n}$.

                       用数归,显然$n=2$时$\frac{1}{2!}+\frac{1}{2!2}=1-\frac{1}{2!1\cdot2}$;

                       假设$n=k$时成立,即$\frac{1}{2!}+\cdots+\frac{1}{k!}+\frac{1}{k!k}=1-\frac{1}{2!1\cdot2}-\cdots-\frac{1}{k!(k-1)k}$,

                       当$n=k+1$时,$\frac{1}{2!}+\cdots+\frac{1}{k!}+\frac{1}{(k+1)!}+\frac{1}{(k+1)!(k+1)}=1-\frac{1}{2!1\cdot2}-\cdots-\frac{1}{k!(k-1)k}+\frac{1}{(k+1)!}+\frac{1}{(k+1)!(k+1)}-\frac{1}{k!k}$.

                       而通分后知$\frac{1}{(k+1)!}+\frac{1}{(k+1)!(k+1)}-\frac{1}{k!k}=-\frac{1}{(k+1)!k(k+1)}$,

                       即$\frac{1}{2!}+\cdots+\frac{1}{k!}+\frac{1}{(k+1)!}+\frac{1}{(k+1)!(k+1)}=1-\frac{1}{2!1\cdot2}-\cdots-\frac{1}{k!(k-1)k}-\frac{1}{(k+1)!k(k+1)}$,也即命题对$n=k+1$也成立,由数归知对所有$n\geqslant2$都成立.
                   \end{solution}

             \item $e=3-\lim_{n\to\infty}\sum_{k=0}^n\frac{1}{(k+2)!(k+1)(k+2)}$;
                   \begin{solution}
                       由(1)知$3-\lim_{n\to\infty}\sum_{k=0}^n\frac{1}{(k+2)!(k+1)(k+2)}=\lim_{n\to\infty}\left(\sum_{k=0}^n\frac{1}{k!}+\frac{1}{n!n}\right)$

                       而前已证$\lim_{n\to\infty}\sum_{k=0}^n\frac{1}{k!}=e$,且显然$\lim_{n\to\infty}\frac{1}{n!n}=0$,由极限的四则运算知等式成立.
                   \end{solution}

             \item 用$\sum_{k=0}^n\frac{1}{k!}+\frac{1}{n!n}$计算$e$要比不加上最后一项好得多.
                   \begin{solution}
                       由(1)的右边可以看出$\lim_{n\to\infty}\left(\sum_{k=0}^n\frac{1}{k!}+\frac{1}{n!n}\right)$是大于$e$的.

                       也就是说,$\lim_{n\to\infty}\sum_{k=0}^n\frac{1}{k!}$单调增从下方逼近$e$,$\lim_{n\to\infty}\left(\sum_{k=0}^n\frac{1}{k!}+\frac{1}{n!n}\right)$单调减从上方逼近$e$.

                       而他们和$e$的误差为$\alpha_n=e-\sum_{k=0}^n\frac{1}{k!}=\sum_{k=n+1}^\infty\frac{1}{k!},\beta_n=\sum_{k=0}^n\frac{1}{k!}+\frac{1}{n!n}-e=\sum_{k=n+1}^\infty\frac{1}{k!(k-1)k}$.

                       显然$\beta_n<\alpha_n$也就是说后者收敛得更快.
                   \end{solution}

         \end{enumerate}

     \end{exercise}

     \begin{exercise}
         设$a_n=1+\frac{1}{\sqrt{2}}+\frac{1}{\sqrt{3}}+\cdots+\frac{1}{\sqrt{n}}-2\sqrt{n},n\in \mathbb{N}_+$,
         证明:$\{a_n\}$收敛.
     \end{exercise}
     \begin{solution}
         $\because\frac{1}{\sqrt{k}}=\frac{2}{2\sqrt{k}}<\frac{2}{\sqrt{k}+\sqrt{k-1}}=2\left(\sqrt{k}-\sqrt{k-1}\right)\Rightarrow a_n<\left(-\sqrt{0}+\sqrt{1}\right)+\cdots+\left(-\sqrt{n-1}+\sqrt{n}\right)-2\sqrt{n}=0$.

         而考虑$a_{n+1}-a_n$,有差值为$\frac{1}{\sqrt{n+1}}+2(\sqrt{n+1}-\sqrt{n})>0$,故$a_n$单调递增有上界,即$\{a_n\}$收敛.
     \end{solution}

     \begin{exercise}
         设已知存在极限$\lim_{n\to\infty}\frac{a_1+a_2+\cdots+a_n}{n}$,证明:$\lim_{n\to\infty}\frac{a_n}{n}=0$.
     \end{exercise}
     \begin{solution}
         (有点取巧?)

         $\lim_{n\to\infty}\frac{a_1+\cdots+a_n}{n}=\lim_{n\to\infty}\frac{a_1+\cdots+a_{n-1}}{n-1}\cdot\frac{n-1}{n}+\frac{a_n}{n}=\lim_{n\to\infty}\frac{a_1+\cdots+a_{n-1}}{n-1}\cdot\frac{n-1}{n}+\lim_{n\to\infty}\frac{a_n}{n}$

         而$\lim_{n\to\infty}\frac{a_1+\cdots+a_{n-1}}{n-1}\cdot\frac{n-1}{n}=\lim_{n\to\infty}\frac{a_1+\cdots+a_n}{n}$

         故$\lim_{n\to\infty}\frac{a_n}{n}=0$.
     \end{solution}

     \begin{exercise}
         证明:$\lim_{n\to\infty}(n!)^{1/n^2}=1$.
     \end{exercise}
     \begin{solution}
         $\sqrt[n]{n!}<\sqrt[n]{n^n}=n$,而前已证$\lim_{n\to\infty}\sqrt[n]{n}=1$,又显然$(n!)^{1/n^2}>1$,由夹逼定理得命题成立.
     \end{solution}

     \begin{exercise}
         设对每个$n$有$x_n<1$和$(1-x_n)x_{n+1}\geqslant\frac{1}{4}$,证明$\{x_n\}$收敛,并求其极限.
     \end{exercise}
     \begin{solution}
         (解法类似2.6练习题7.)

         先证$n\geqslant2$时$x_n$都为正数.

         因为$x_n<1$,所以$(1-x_n)>0$恒成立,故$x_{n+1}>0$,也就是说从$n=2$开始$\{x_n\}$都为正数.

         再证$\{x_n\}\leqslant\frac{1}{2}$,

         $(1-x_{n-1})x_n\geqslant\frac{1}{4}\Rightarrow x_n\geqslant\frac{1}{4(1-x_{n-1})}$,但$x_n<1$,所以$\frac{1}{4(1-x_{n-1})}<1\Rightarrow x_{n-1}<\frac{3}{4}$;同理$x_{n-1}<\frac{3}{4}\Rightarrow x_{n-2}<\frac{2}{3}\cdots$,若$x_n<m$,则$x_{n-1}<1-\frac{1}{4m}$,易证这是个单调减趋向$\frac{1}{2}$的序列,令n趋向无穷便得到$\{x_n\}$每一项都$\leqslant\frac{1}{2}$.

         从下方夹逼,

         $x_1<1\Rightarrow(1-x_1)>0\Rightarrow x_2>0\Rightarrow(1-x_2)<1\Rightarrow x_3>\frac{1}{4}\Rightarrow x_4>\frac{1}{3}\cdots$,同样可证这是一个单调增趋向$\frac{1}{2}$的序列,由夹逼定理知$\{x_n\}$收敛且极限为$\frac{1}{2}$.
     \end{solution}

     \begin{exercise}
         设$a_1=b,a_2=c$,再$n\geqslant3$时,$a_n=\frac{a_{n-1}+a_{n-2}}{2}$,证明$\{a_n\}$收敛,并求其极限.
     \end{exercise}
     \begin{solution}
         一个方法是用特征根法解出通项为$a_n=(\frac{1}{3}b+\frac{2}{3}c)+\frac{4}{3}(c-b)(-\frac{1}{2})^n$,令n趋向无穷便得到极限为$\frac{1}{3}b+\frac{2}{3}c$.

         另外也可以像提示中那样证明奇数项和偶数项分别单调互为上下界.
     \end{solution}

     \begin{exercise}
         设$a,b,c$是三个给定的实数,令$a_1=a,b_1=b,c_1=c$,并以递推公式定义
         \[
             a_{n+1}=\frac{b_n+c_n}{2},b_{n+1}=\frac{a_n+c_n}{2},c_{n+1}=\frac{a_n+b_n}{2},n\in \mathbb{N}+
         \]
         .
         求这三个数列的极限.
     \end{exercise}
     \begin{solution}
         注意到有$a_n+b_n+c_n=a+b+c$,且$a_{n+1}-b_{n+1}=-\frac{1}{2}(a_n-b_n)\Rrightarrow a_n-b_n=\left(-\frac{1}{2}\right)^n(a-b)$,同理对$b_n,c_n$也可以得到类似的式子,联立可以解出$\{a_n\},\{b_n\},\{c_n\}$,最后可以得到极限为$\frac{a+b+c}{3}$.
     \end{solution}

     \begin{exercise}
         \begin{enumerate}
             \item 设$a_1>b_1>0,a_{n+1}=\frac{2a_nb_n}{a_n+b_n},b_{n+1}=\sqrt{a_{n+1}b_n},n\in \mathbb{N}+$,证明:
                   $\{a_n\}$和$\{b_n\}$收敛于同一极限.

             \item 在$a_1=2\sqrt{3},b_1=3$时,证明上述极限等于单位圆的半周长$\pi$.(这里可以利用极限$\lim_{n\to\infty}n\sin\frac{\pi}{n}=\pi.)$
         \end{enumerate}
     \end{exercise}

     \begin{solution}
         \begin{enumerate}
             \item 解法类似例题2.3.5,分别证明$\{a_n\},\{b_n\}$单调且互为上下界.

                   $a_n>b_n>0\Rightarrow a_n+b_n>2b_n\Rightarrow 1>\frac{2b_n}{a_n+b_n}\Rightarrow a_n>\frac{2a_nb_n}{a_n+b_n}$

                   $a_n>b_n>0\Rightarrow 2a_n>a_n+b_n\Rightarrow a_{n+1}=\frac{2a_nb_n}{a_n+b_n}>b_n\Rightarrow b_{n+1}=\sqrt{a_{n+1}b_n}>b_n$

                   $a_n>b_n\Rightarrow a_{n+1}>\sqrt{a_{n+1}b_n}=b_{n+1}$

                   故两者都单调有界,也即收敛.

                   任取其中一个递推式,令$n$趋向无穷即得两者极限相同.
             \item 若$\{a_n\}$为$m$边单位圆外切正多边形半周长,$\{b_n\}$为$m$边单位圆内接正多边形半周长,则$a_n=m\tan\frac{2\pi}{2m},b_n=m\sin\frac{2\pi}{2m}$.

                   验证知$\frac{2a_nb_n}{a_n+b_n}=a_{n+1}=2m\tan\frac{2\pi}{2(2m)},\sqrt{a_{n+1}b_n}=b_{n+1}=2m\sin\frac{2\pi}{2(2m)}$,即下一项为$2m$边正多边形的半周长.

                   计算知$a_1,b_1$分别为正六边形外切(内接)圆半周长,故$a_n=6\cdot2^{n-1}\tan\frac{\pi}{6\cdot2^{n-1}},$

                   $b_n=6\cdot2^{n-1}\sin\frac{\pi}{6\cdot2^{n-1}}$,由提示知极限为$\pi$.
         \end{enumerate}
     \end{solution}
     \begin{note}
         本题与例题2.3.5完全不同.实际上这就是计算圆周率的Archimedes(阿基米德)-刘徽方法的迭代形式.在(2)中的两个数列$\{a_n\}$和$\{b_n\}$就是单位圆的外切和内接正多边形的半周长(请求出边数和$n$的关系).
     \end{note}

 \subsection{第二组}
     \begin{exercise}
         设$a_n = \sqrt{1+\sqrt{2+\cdots +\sqrt{n}}}$, $n\in \mathbb{N}_+$, 证明:$\{a_n\}$收敛.
     \end{exercise}
     \begin{solution}
         根据提示, 有$\sqrt{n-1+\sqrt{n}}<\sqrt{n-1+2\sqrt{n-1}+1}=\sqrt{n-1}+1$,再利用$\sqrt{n-1}<2\sqrt{n-2}(n\geqslant 3)$逐层脱去根号, 最后有$\leqslant2+\sqrt{2}$
     \end{solution}

     \begin{exercise}
         证明: 对每个正整数$n$, 成立不等式$\left(1+\frac{1}{n}\right)^n>\sum_{k=0}^n \frac{1}{k!}-\frac{e}{2n}$.
     \end{exercise}
     \begin{solution}
         这里怀疑提示有点问题, 应当为利用$(1-\frac{1}{n})(1-\frac{2}{n})\cdots\geqslant 1-\frac{1}{n}-\frac{2}{n}\cdots$.

         将左边展开, 有$\left(1+\frac{1}{n}\right)^n=1+n\frac{1}{n}+\frac{n(n-1)}{2!}\frac{1}{n^2}+\cdots=1+1+\frac{1\cdot (1-\frac{1}{n})}{2!}+\frac{1\cdot (1-\frac{1}{n})(1-\frac{2}{n})}{3!}\cdots$.

         利用伯努利不等式有$(1-\frac{1}{n})(1-\frac{2}{n})\cdots\geqslant 1-\frac{1}{n}-\frac{2}{n}\cdots=1-\frac{\frac{n(n-1)}{2}}{n}$, 将不等式右边移项, 则只需证$-\frac{1}{2!}\frac{2(2-1)}{2n}-\frac{1}{3!}\frac{3(3-1)}{2n}-\cdots\geqslant-\frac{e}{2n}\Longleftrightarrow\frac{1}{0!}+\frac{1}{1!}+\cdots\leqslant e$, 而这正是我们已经知道的结果.
     \end{solution}

     \begin{exercise}
         求极限$\lim_{n\to\infty}n\sin(2\pi n!e)$.
     \end{exercise}
     \begin{solution}
         注意到$n!\frac{1}{k!}$是整数, 利用三角函数的周期性有$n\sin(2\pi n!e)=n\sin[2\pi n!(e-1-1-\frac{1}{2!}-\cdots)]$, 利用2.5.4的误差估计就可以得到$n\sin\frac{2\pi}{n+1}<n\sin(2\pi n!e)<n\sin\frac{2\pi}{n}$, 从夹逼以及三角极限可知答案为$2\pi$.
     \end{solution}

     \begin{exercise}
         记$S_n=1+\frac{1}{2}+\cdots+\frac{1}{n}, n\in\mathbb{N}_+$. 用$K_n$表示使$S_k\geqslant n$的最小下标, 求极限$\lim_{n\to\infty}\frac{K_{n+1}}{K_n}$.
     \end{exercise}
     \begin{solution}
         调和级数近似于$\ln n$, 所以对应的$K_n$应当近似于$e^n$,所以答案明显.
     \end{solution}

     \begin{exercise}
         设$x_n=\frac{1}{n^2}\sum_{k=0}^n\ln\binom{n}{k} ,n\in\mathbb{N}_+$, 求$\lim_{n\to\infty}x_n$.
     \end{exercise}
     \begin{solution}
         用两次Stolz定理,
         \begin{align*}
             \lim_{n\to\infty}x_n
              & =\lim_{n\to\infty}\frac{1}{n^2}[(n+1)(\ln 1+\ln 2+\cdots)-2((\ln 1)+(\ln 1+\ln 2)+\cdots)]                \\
              & =\lim_{n\to\infty}\frac{1}{2n+1}[(\ln 1+\ln 2+\cdots+\ln n)+(n+2)\ln(n+1)-2(\ln 1+\ln 2+\cdots+\ln(n+1))] \\
              & =\lim_{n\to\infty}\frac{1}{2n+1}[(n+1)\ln(n+1)-(\ln 1+\ln 2+\cdots+\ln(n+1))]                             \\
              & =\lim_{n\to\infty}\frac{n}{2}\ln (1+\frac{1}{n})
         \end{align*}
         答案是$\frac{1}{2}$
     \end{solution}

     \begin{exercise}
         将二项式系数$\binom{n}{0},\binom{n}{1},\cdots,\binom{n}{n}$的算数平均值和几何平均值分别记为$A_n$和$G_n$. 证明:
         \begin{enumerate}
             \item $\lim_{n\to\infty}\sqrt[n]{A_n}=2$,
             \item $\lim_{n\to\infty}\sqrt[n]{G_n}=\sqrt{e}$.
         \end{enumerate}
     \end{exercise}
     \begin{solution}
         \begin{enumerate}
             \item 利用$\sum\binom{n}{k}=2^n$, $\lim_{n\to\infty}\sqrt[n]{A_n}=\frac{2}{\displaystyle\lim_{n\to\infty}\sqrt[n]{n}}=2$.
             \item 对待求式取对数就变成了上一题, 利用结论即可.
         \end{enumerate}
     \end{solution}

     \begin{exercise}
         设$A_n=\sum_{k=1}^{n}a_k, n\in \mathbb{N}_+$, 数列$\{A_n\}$收敛. 又有一个单调增加的正数数列$\{p_n\}$, 且为正无穷大亮. 证明: $\lim_{n\to\infty}\frac{p_1a_1+p_2a_2+\cdots+p_na_n}{p_n}=0$.
     \end{exercise}
     \begin{solution}
         类似于Abel变换, 将$a_n$改写成$A_n-A_{n-1}$,
         \begin{align*}
             \text{原式}
              & = \lim_{n\to\infty}\frac{p_1A_1+p_2(A_2-A_1)+\cdots+p_{n-1}(A_{n-1}-A_{n-1})+p_n(A_n-A_{n-1})}{p_n} \\
              & = \lim_{n\to\infty}\frac{A_1(p_1-p_2)+\cdots+A_{n-1}(p_{n-1}-p_n)+A_np_n}{p_n}
         \end{align*}
         从中间某处截断, 有:
         \begin{align*}
             \text{原式}
              & = \lim_{n\to\infty}\frac{A_1(p_1-p_2)+\cdots+A_k(p_k-p_{k+1})}{p_n}+\frac{A_{k+1}(p_{k+1}-p_{k+2})+\cdots+A_{n-1}(p_{n-1}-p_n)+A_np_n}{p_n}
         \end{align*}
         容易看出左边一部分极限趋0, 而不妨设$\{A_n\}\to a,(a>0)$, 则右边有$\frac{(a+\varepsilon)(p_{k+1}-p_n)+A_np_n}{p_n}\leqslant \text{右式}\leqslant \frac{(a-\varepsilon)(p_{k+1}-p_n)+A_np_n}{p_n}$.

         由夹逼定理同样易知极限为0, 故待求极限为0.
     \end{solution}

     \begin{exercise}
         设$\{a_n\}$满足$\lim_{n\to\infty}(a_n \sum_{i=1}^{n}a_i^2)=1$, 证明:$\lim_{n\to\infty}\sqrt[3]{3n}a_n=1$.
     \end{exercise}
     \begin{solution}

         首先证明$\sum_{i=1}^{n}a_i^2$趋向正无穷大.

         符号显然, 且容易看出单调递增, 由单调数列的性质知只可能有极限或者趋向正无穷大.

         反证, 如果存在极限$a$, 则由题设知$a>0$. 由极限的四则运算知$\{a_n\}$趋向$\frac{1}{a}$; 然而这与$\sum_{i=1}^{n}a_i^2$存在极限矛盾, 故$\sum_{i=1}^{n}a_i^2$趋向正无穷大.一个推论就是$a_n^2$趋0, 也即$\lim_{n\to\infty} a_n=0$.

         再证$a_n^3\thicksim \frac{1}{n}$.

         由Stolz知有$\lim_{n\to\infty}\frac{a_{n+1}^2}{\frac{1}{a_{n+1}}-\frac{1}{a_n}}=1$, 也即$\lim_{n\to\infty}\frac{a_{n+1}^3}{1-\frac{a_{n+1}}{a_n}}=1$

         对于题设, 做除法有$\lim_{n\to\infty}\frac{a_{n+1}\sum_{i=1}^{n+1}a_i^2}{a_n \sum_{i=1}^{n}a_i^2}=\lim_{n\to\infty}\frac{a_{n+1}}{a_n}(1+\frac{a_{n+1}^2}{\sum_{i=1}^{n}a_i^2})=1$, 显然中间式子括号部分趋向于1, 也即$\lim_{n\to\infty}\frac{a_{n+1}}{a_n}=1$.

         结合知有$\lim_{n\to\infty}\frac{a_{n+1}^3}{(1-\frac{a_{n+1}}{a_n})(1+\frac{a_{n+1}}{a_n}+(\frac{a_{n+1}}{a_n})^2)}=\lim_{n\to\infty}\frac{a_{n+1}^3}{1-(\frac{a_{n+1}}{a_n})^3}=\lim_{n\to\infty}\frac{1}{\frac{1}{a_{n+1}^3}-\frac{1}{a_n^3}}=\frac{1}{3}$

         故由Stolz知$\lim_{n\to\infty}\frac{3n}{a_n^3}=1$.
     \end{solution}
     \begin{note}
         感觉要一开始就看出$a_n^3\thicksim \frac{1}{n}$还是很有难度的, 只能从待求结论得到提示...
     \end{note}

     \begin{exercise}
         设数列$\{u_n\}_{n \geqslant 0}$对每个非负整数$n$满足条件
         \[
             u_n=\lim_{m\to\infty}(u_{n+1}^2+u_{n+2}^2+\cdots+u_{n+m}^2),
         \]
         证明:若存在有限极限$\lim_{n\to\infty}(u_1+u_2+\cdots+u_n)$,则只能是每个$u_n=0$.
     \end{exercise}
     \begin{solution}
         显然$\lim_{n\to\infty}u_n=0$.

         首先证明$\{u_n\}$的一个性质:$\{u_n\}$要么全0要么全部非负.

         如果存在某一项$u_k$为0, 则根据题设知后边项全部为0; 同时由于$u_{k-1}=u_k^2+\cdots$, 故递推知前边所有项也都为0.

         再证如果$u_n$不全为0则$u_n\thicksim\frac{1}{n}$, 即$\lim_{n\to\infty}nu_n=1$.

         首先分析$\lim_{n\to\infty}\frac{1}{\frac{1}{u_{n+1}}+\frac{1}{u_n}}$,稍作变换得
         \[
             =\lim_{n\to\infty}\frac{u_{n+1}u_n}{u_n-u_{n+1}}
         \]
         由题设可以得到$u_n=u_{n+1}^2+u_{n+1}\Longleftrightarrow u_n-u_{n+1}=u_{n+1}^2$, 代入得
         \begin{align*}
              & =\lim_{n\to\infty}\frac{u_n}{u_{n+1}}               \\
              & =\lim_{n\to\infty}\frac{u_{n+1}^2+u_{n+1}}{u_{n+1}} \\
              & =\lim_{n\to\infty}u_n+1                             \\
              & =1
         \end{align*}
         这时结论就很明显了, 在$u_n$和$\frac{1}{n}$等价的同时, 可以知道$\lim_{n\to\infty}\sum_{i=1}^{n}u_i$趋向于无穷大, 要得到有限极限只可能是每一项都为0.
     \end{solution}
     \begin{note}
         同样地要看出$a_n\thicksim \frac{1}{n}$太难了...
     \end{note}

     \begin{exercise}
         (Toeplitz定理)设对$n,k\in \mathbb{N}_+$,有$t_{nk}\geqslant 0$. 又有$\sum_{k=1}^{n}t_{nk}=1, \lim_{n\to\infty}t_{nk}=0$. 若已知$\lim_{n\to\infty}a_n=a$, 定义$x_n=\sum_{k=1}^{n}t_{nk}a_k, n\in \mathbb{N}_+$. 证明:$\lim_{n\to\infty}x_n=a$.

         几种变形:
         \begin{enumerate}
             \item 将条件$\sum_{k=1}^{n}t_{nk}a_k$改为$\lim_{n\to\infty}\sum_{k=1}^{n}t_{nk}=1$;
             \item 不要求$t_{nk}$非负, 将(1)中的条件改为存在$M>0$, 使得对每个$n$, 成立不等式$|t_{n1}|+\cdots+|t_{nn}|\leqslant M$. 则结论对$a=0$仍成立.
         \end{enumerate}
     \end{exercise}
     \begin{solution}
         类似于Cauthy, 将原式分割成两部分:$(t_{n1}a_1+\cdots t_{nm}a_m)+(t_{n(m+1)}+a_{m+1}+\cdots+t_{nn}a_n)$.

         由题设, 对于任意$\varepsilon>0$, 可以找到适当的$N,m$, 使得当$n>N$时成立$|a_i-a|<\frac{\varepsilon}{2}, m<i \leqslant n$, 同时$\sum_{k=1}^{m}t_{nk}a_k<\frac{\varepsilon}{2}$(方法是首先找到适当的m让之后的$a_n$都符合要求, 再将$n$拉长使得前$m$个和符合要求), 这时成立$x_n<\varepsilon, n>N$, 不断缩小$\varepsilon$就可以证明结论.
     \end{solution}
     \begin{note}
         \begin{enumerate}
             \item 这时只要对分割开的后半部分引入另一个小量$\eta $即可. 先一步变成$\sum_{k=m+1}^{n}t_{nk}(a-\frac{\varepsilon}{2})\leqslant \sum_{k=m+1}^{n}t_{nk}a_k \leqslant \sum_{k=m+1}^{n}t_{nk}(a+\frac{\varepsilon}{2})$, 引入$\eta$后最左最有再添加两项变成$(1-\eta)(a-\frac{\varepsilon}{2})\leqslant \cdots \leqslant (1+\eta)(a+\frac{\varepsilon}{2})$, 这样就夹住了结论.
             \item 这时就变化成$-M\frac{\varepsilon}{2}\leqslant \cdots \leqslant M\frac{\varepsilon}{2}$
         \end{enumerate}
     \end{note}

     \begin{exercise}
         用Toeplitz定理导出Stolz定理.
     \end{exercise}
     \begin{solution}
         不妨设$a_0=b_0=0$,同时令$t_{nk}=\frac{a_k-a_{k-1}}{a_n}$,则当$\lim_{n\to\infty}\frac{b_{n+1)-b_n}}{a_{n+1}-a_n}=l$时根据Toeplitz定理有$\sum_{k=1}^{n}t_{nk}\frac{b_k-b_{k-1}}{a_k-a_{k-1}}=\sum_{k=1}^{n}\frac{a_k-a_{k-1}}{a_n}\frac{b_k-b_{k-1}}{a_k-a_{k-1}}=\frac{b_{n}}{a_{n}}=l$.
     \end{solution}
     \begin{note}
         似乎无法导出$\frac{0}{0}$型的Stolz.
     \end{note}

     \begin{exercise}
         设$0<\lambda<1,\{a_n\}$收敛于$a$. 证明:
         \[
             \lim_{n\to\infty}(a_n+\lambda a_{n-1}+{\lambda}^2 a_{n-2}+\cdots+{\lambda}^n a_0)=\frac{a}{1-\lambda}.
         \]
     \end{exercise}
     \begin{solution}
         由于$\{a_n\}\to a$, 由收敛性质知$\{a_n\}$有界, 即有$|a_n|<A, \forall\, n\in \mathbb{N}_+$, 又知存在$N$使得$|\lambda^{N+1}a_{N+1}+\cdots|\leqslant \lambda^{N+1}A+\cdots \leqslant \frac{\lambda^{N+1}}{1-\lambda}A<\frac{\varepsilon}{2},\varepsilon$为任意小量; 由于极限性质同时存在$N'$满足$|a_n-a|<\frac{\varepsilon}{2}, n>N'$, 易知足够大的$N$可以同时满足两个要求, 这时将待求式从左N处分开, 即求:
         \[
             \lim_{n\to\infty}(a_n+\lambda^2 a_{n-1}+\cdots+\lambda^N a_k)+(\lambda^{N+1}a_{k-1}+\cdots+\lambda_n a_0)(k>N)
         \]
         易见右半部分极限趋向0, 而左半部分有$\frac{(a-\varepsilon)(1-\lambda^{N+1})}{1-\lambda}\leqslant \cdots \leqslant \frac{(a+\varepsilon)(1-\lambda^{N+1})}{1-\lambda}$令$\varepsilon\to 0, N\to\infty$知左半部分趋$a$, 证毕.
     \end{solution}

     \begin{exercise}
         设$\lim_{n\to\infty}x_n=0$, 并且存在常数$K$, 使得$|y_1|+|y_2|+\cdots+|y_n|\leqslant K$对每个$n$成立. 令$z_n=x_1y_n+x_2y_{n-1}+\cdots+x_ny_1, n\in \mathbb{N}_+$, 证明: $\lim_{n\to\infty}z_n=0$.
     \end{exercise}
     \begin{solution}
         同样也是划分成两个部分, 前一部分由于$y_n\to 0$可以变得任意小; 后半部分因为$x_n$可以变得任意小, 同时$\{y_n\}$前边加起来小于$K$, 也即可以任意小, 故整体极限为0.
     \end{solution}
     \begin{note}
         从本题条件已可推出$\lim_{n\to\infty}y_n=0$. 但是可以举出例子说明仅有条件$\lim_{n\to\infty}x_n=\lim_{n\to\infty}y_n=0$不能得到$\lim_{n\to\infty}z_n=0$.
     \end{note}
     \begin{note}
         $y_n\to 0$十分显然, 严格来说即: 令$Y_n=|y_1|+|y_2|+\cdots+|y_n|$, 由题设知这是单调递增有上界数列, 故有极限存在, 同时$\lim_{n\to\infty}Y_n-Y_{n-1}=0$, 也即$|y_n|\to 0$.

         根据提示, 取$x_n=y_n=\frac{1}{\sqrt{n}}$, 则$z_n=\frac{1}{\sqrt{1}}\cdot\frac{1}{\sqrt{n}}+\frac{1}{\sqrt{2}}\cdot\frac{1}{\sqrt{n-1}}+\cdots+\frac{1}{\sqrt{n}}\cdot\frac{1}{\sqrt{1}}$.

         简单的利用均值不等式我们可以得到$z_n \geqslant n\sqrt[n]{\frac{1}{n!}}$, 而根据前边的结论我们知道$\lim_{n\to\infty}\sqrt[n]{\frac{n^n}{n!}}=e$(见例题2.7.2的注5, 证明方法是利用$a_n\to a \Rightarrow \sqrt[n]{a_1a_2\cdots a_n}=a(\forall\, n \in \mathbb{N}_+, a_n>0)$), 可以得到$z_n$绝不趋向0.
     \end{note}

     \begin{exercise}
         设$y_n=2x_{n+1}+x_n, n\in \mathbb{N}_+$. 证明: 若$\{y_n\}$收敛, 则$\{x_n\}$也收敛.
     \end{exercise}
     \begin{solution}
         将$x_n$前几项写出, 有:
         \begin{align*}
             x_1 & =x_1                                                               \\
             x_2 & =\frac{1}{2}(y_1-x_1)=\frac{1}{2}y_1-\frac{1}{2}x_1                \\
             x_3 & =\frac{1}{2}(y_2-x_2)=\frac{1}{2}y_2-\frac{1}{4}y_1+\frac{1}{4}x_1 \\
             \vdots
         \end{align*}
         可以用数学归纳法证明有$x_n=\frac{1}{2}y_{n-1}-\frac{1}{4}y_{n-2}+\cdots-(-\frac{1}{2})^{n-1}y_1+(-\frac{1}{2})^{n-1}x_1$,则利用12题的结论可以得出$\lim_{n\to\infty}x_n=\frac{1}{2}(\frac{Y}{1-(-\frac{1}{2})})=\frac{1}{3}Y$.
     \end{solution}

     \begin{exercise}
         由初始值$a_0$和$a_n=2^{n-1}-3a_{n-1}, n \in \mathbb{N}_+$, 确定数列$\{a_n\}$. 求$a_0$的所有可能值, 使得数列$\{a_n\}$是单调增加的.
     \end{exercise}
     \begin{solution}
         我也忘记这叫啥方法了, 构造辅助数列的时候要让辅助数列中n在指数上的项跟着动. 具体来说, 在这里需要求解$a_n+\lambda 2^n=-3(a_{n-1}+\lambda 2^{n-1})$, 解得$\lambda=-\frac{1}{5}$, 也即$a_n=(-3)^n(a_0-\frac{1}{5})+\frac{1}{5}2^n$.

         可以体会出由于前一项的底数绝对值更大, 起主导作用, 最后整个符号随着前一项摆动, 要使$\{a_n\}$严格单调递增, 只能是前一项的系数为0, 也即$a_0=\frac{1}{5}$.
     \end{solution}

     \begin{exercise}
         证明数列$\sqrt{7}, \sqrt{7-\sqrt{7}}, \sqrt{7-\sqrt{7+\sqrt{7}}}, \sqrt{7-\sqrt{7+\sqrt{7-\sqrt{7}}}}, \cdots$收敛, 并求其极限.
     \end{exercise}
     \begin{solution}
         首先发现隔项有递推式$a_n=\sqrt{7-\sqrt{7+a_{n-2}}}$.

         之后列出不动点方程: $x^4-14x^2-x+42=(x-2)(x^3+2x^2-10x-21)=0$, 这个2是猜根猜出来的.(可以尝试把常数项分解因子, 然后挨个带进去)(利用数学软件算出还有个好根-3, 另两个根是$1\pm \sqrt{29}$)

         观察原来的表达式, 首先都是根号下说明都是正数, 同时从第二项开始为$\sqrt{7-\cdots}$说明后边的项都不超过$\sqrt{7}$, 很有可能2就是极限.

         两边同时减去2, 整理得到$a_n-2=\frac{2-a_{n-2}}{(\sqrt{7-\sqrt{7+a_{n-2}}}+2)(3+\sqrt{7+a_{n-2}})}$, 可以看出分母不小于6, 也即$a_n-2 \leqslant \frac{1}{6}(2-a_{n-2})$, 所以得到极限为2.
     \end{solution}

     \begin{exercise}
         令$y_0 \geqslant 2, y_n=y_{n-1}^2-2, n \in \mathbb{N}_+$. 设$S_n=\frac{1}{y_0}+\frac{1}{y_0y_1}+\cdots+\frac{1}{y_0y_1\cdots y_n}$. 证明: $\lim_{n\to\infty}S_n=\frac{y_0-\sqrt{y_0^2-4}}{2}$.
     \end{exercise}
     \begin{solution}
         根据提示, 做代换$y_0=x+\frac{1}{x}$, 则有$y_n=x^{2^n}+\frac{1}{x^{2^n}}$;

         分母简单同分之后, 对分子写出前几项有:
         \begin{align*}
             s_1 & =x^2+1+\frac{1}{x^2}                                                                                                                               \\
             s_2 & =x^6+x^4+x^2+1+\frac{1}{x^2}+\frac{1}{x^4}+\frac{1}{x^6}                                                                                           \\
             s_3 & =x^{14}+x^{12}+x^{10}+x^8+x^6+x^4+x^2+1+\frac{1}{x^2}+\frac{1}{x^4}+\frac{1}{x^6}+\frac{1}{x^8}+\frac{1}{x^{10}}+\frac{1}{x^{12}}+\frac{1}{x^{14}} \\
             \vdots
         \end{align*}
         观察之后可以发现, 分子有递推式$s_n=s_{n-1}(x^n+\frac{1}{x^n})+1$, 恰恰好括号中的两项分别将$s_{n-1}$平移到1的左右两边, 最后再补上一个1恢复规律.

         利用归纳法可以证明$s_n=\frac{x^{2^{n+1}}-\frac{1}{x^{2^{n+1}-2}}}{x^2-1}$, 而原分式的分母有$\frac{1}{y_0y_1\cdots y_n}=\frac{x-\frac{1}{x}}{x^{2^{n+1}}-\frac{1}{x^{2^{n+1}}}}$;

         故$S_n=\frac{x^{2^{n+1}-1}-\frac{1}{x^{2^{n+1}-1}}}{x^{2^{n+1}}-\frac{1}{x^{2^{n+1}}}}=\frac{\frac{1}{x}-\frac{1}{x^{2^{n+1}-1}x^{2^{n+1}}}}{1-\frac{1}{x^{2^{2n+2}}}}$, 不妨设$x>1$, 容易看出$S_n\to \frac{1}{x}$, 也即$\lim_{n\to\infty}S_n=\frac{y_0-\sqrt{y_0^2-4}}{2}$.
     \end{solution}

     \begin{exercise}
         设$x_1=c, x_{n+1}=a^{x_n}(a>0, a\neq 1), n \in \mathbb{N}_+$. 根据下面提供的函数$f(x)=a^x$和$f(f(x))$的单调性和不动点的知识, 讨论数列$\{x_n\}$的敛散性.
         \begin{enumerate}
             \item 在$a>1$时函数$f(x)=a^x$单调增加.
                   \begin{enumerate}
                       \item 如$a>e^\frac{1}{e}$, 则$f$无不动点. 证明: 不论$c$如何, 数列$\{x_n\}$总是单调增加的正无穷大量;
                       \item 在$a>e^\frac{1}{e}$时$f$恰有一个不动点. 证明: 当$c \leqslant e$时, 数列$\{x_n\}$单调增加收敛于$e$, 而当$c>e$时, $\{x_n\}$是单调增加的正无穷大量.
                   \end{enumerate}
         \end{enumerate}
     \end{exercise}
