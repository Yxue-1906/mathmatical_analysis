
\section{Cauthy收敛准则}

 \begin{definition}{基本数列}{Cauchy Sequence}
     称数列$\{x_n\}$为基本数列(或Cauthy数列),如果对每个$\varepsilon>0$,存在$N$,使得对每一对正整数$n,m>N$,成立估计式$\lvert a_n-a_m\rvert<\varepsilon$.
 \end{definition}

 \begin{theorem}{Cauthy收敛准则}{Cauthy Theorem}
     数列收敛的充分必要条件是该数列为基本数列.
 \end{theorem}

 \begin{definition}{压缩映射}{Contraction Mapping}
     设函数$f$在区间$[a,b]$上定义,$f([a,b])\subset [a,b]$,并存在一个常数$k$,满足$0<k<1$,使得对一切 $x,y\in [a,b]$成立不等式$\lvert f(x)-f(y)\rvert\leqslant k\lvert x-y\rvert$,则称$f$是$[a,b]$上的一个压缩映射,称常数$k$为压缩常数.
 \end{definition}
 \begin{theorem}{压缩映射原理}{Contraction Mapping Theorem}
     设$f$是$[a,b]$上的一个压缩映射,则
     \begin{enumerate}
         \item $f$在$[a,b]$中存在唯一的不动点$\xi=f(\xi)$;
         \item 由任何初始值$a_0\in[a,b]$和递推公式$a_{n+1}=f(a_n),n\in \mathbb{N}_+$生成的数列$\{a_n\}$一定收敛于$\xi$;
         \item 成立估计式$\lvert a_n-\xi\rvert\leqslant\frac{k}{1-k}\lvert a_n-a{n-1}\rvert$和$\lvert a_n-\xi\rvert\leqslant\frac{k^n}{1-k}\lvert a_1-a_0\rvert$(即事后估计与先验估计).
     \end{enumerate}
 \end{theorem}
 \subsection{思考题}
     \begin{example}
         Cauthy收敛准则在有理数集$\mathbb{Q}$中不不成立.
     \end{example}
     \begin{solution}
         原因和上一节相同,都是可能存在在实数系中的极限但是不在有理数集中.

         这个\href{https://www.zhihu.com/question/50995932/answer/866173110}{链接}有更加专业的解释,可以参考下图.

         \includegraphics[width=0.8\linewidth]{Picture/3.4/Cauthy收敛准则在Q中不成立.png}
     \end{solution}

 \subsection{练习题}
     \begin{exercise}
         满足以下条件的数列$\{x_n\}$是否一定是基本数列?若回答"是",请做出证明;若回答"不一定是",请举出反例:
         \begin{enumerate}
             \item 对每个$\varepsilon>0$,存在$N$,当$n>N$是,成立
                   $\lvert x_n-x_N\rvert<\varepsilon$;
                   \begin{solution}
                       是.对于任意$\varepsilon>0$,由题设知可以得到$N$,使得当$n>N$时成立$\lvert x_n-x_N\rvert<\frac{\varepsilon}{2},\lvert x_m-x_N\rvert<\frac{\varepsilon}{2}$,则$\lvert x_n-x_m\rvert<\lvert x_n-x_N\rvert+\lvert x_m-x_N\rvert<\varepsilon$.
                   \end{solution}
             \item 对所有$n,p\in \mathbb{N}_+$,成立不等式$\lvert x_{n+p}-x_n\rvert\leqslant\frac{p}{n}$;
                   \begin{solution}
                       不一定是.显然基本数列满足这个条件,但例如$x_n=\sum_{i=1}^{n}\frac{1}{i}$,显然成立题设条件,甚至我们可以直接得到估计,但是我们知道$\lim_{n\to\infty}x_n=+\infty$,所以这不一定是基本数列.
                   \end{solution}
             \item 对所有$n,p\in \mathbb{N}_+$,成立不等式$\lvert x_{n+p}-x_n\rvert\leqslant\frac{p}{n^2}$;
                   \begin{solution}
                       是.根据题设,我们可以得到更精确的估计:$\lvert x_{n+p}-x_n\rvert<\lvert x_{n+p}-x_{n+p-1}\rvert+\cdots+\lvert x_{n+1}-x_{n}\rvert\leqslant\sum_{i=n}^{n+p-1}\frac{1}{i^2}$.根据我们之前得到的结论,$S_n=\sum_{i=1}^{\infty}\frac{1}{i^2}$收敛,所以只需$N$足够大就可以得到$\lvert x_{n+p}-x_n\rvert<S_{n+p}-S_n<\varepsilon$.
                   \end{solution}
             \item 对每个正整数$p$,成立$\lim_{n\to\infty}(x_n-x_{n+p})=0$.
                   \begin{solution}
                       不一定是.例如(2)中的例子$x_n=\sum_{i=1}^{n}\frac{1}{i}$,对每个固定的$p$都满足题设,但它不是基本数列.
                   \end{solution}
         \end{enumerate}
     \end{exercise}

     \hypertarget{2.4.p.2}{}
     \begin{exercise}
         用对偶法则于数列收敛的Cauthy收敛准则,以正面方式写出数列发散的充分必要条件.
     \end{exercise}
     \begin{solution}

         Cauthy收敛准则:$\forall\, \varepsilon>0,\exists\, N\in \mathbb{N}_+$,当$ n,m>N\rightarrow\lvert x_n-x_m\rvert<\varepsilon\Longleftrightarrow \exists\, x\in R,\lim_{n\to\infty}x_n=x$;

         数列发散充要条件:数列$\{x_n\}$发散$\Longleftrightarrow \exists\, \varepsilon>0,\forall\, N\in \mathbb{N}_+,\exists\, n,m>N,\lvert x_n-x_m\rvert\geqslant \varepsilon$.
     \end{solution}

     \begin{exercise}
         证明下列数列为基本数列,因此都是收敛数列:
         \begin{enumerate}
             \item $a_n=1+\frac{1}{2!}+\frac{1}{3!}+\cdots+\frac{1}{n!},n\in \mathbb{N}_+$;
                   \begin{solution}
                       不妨设$m\geqslant n>0$,而显然$\lvert a_m-a_n\rvert=\frac{1}{(n+1)!}+\cdots+\frac{1}{m!}<\frac{1}{2^{n+1}}+\cdots+\frac{1}{2^m}$,而后者是收敛的,知对任意$\varepsilon$都可以找到题目需要的$N$.
                   \end{solution}
             \item $b_n=1-\frac{1}{2}+\frac{1}{3}-\cdots+(-1)^{n+1}\frac{1}{n},n\in \mathbb{N}_+$;
                   \begin{solution}
                       $\lvert x_m-x_n\rvert=\left|(-1)^{n+2}\frac{1}{n+1}+\cdots+(-1)^{m+1}\frac{1}{m}\right|$,若$m-n$为奇数,则知从第二项开始可以每两项凑成一对,而且他们都与第一项正负相反;同时$\lvert x_m-x_n\rvert$与第一项同正负(因为最后一项和第一项同正负,而去掉第一项之后又可以两两凑成一对,这样的对子和第一项同正负),即$\big\lvert (-1)^{n+2}\frac{1}{n+1}+\cdots+(-1)^{m+1}\frac{1}{m}\big\rvert<\lvert \frac{1}{n+1}\rvert$.

                       例如$\left|1-\frac{1}{2}+\frac{1}{3}\right|=\underbrace{\left|(1-\frac{1}{2})+\frac{1}{3}\right|}_{\text{可以看出此式与1同号}}=\underbrace{\left|1+(-\frac{1}{2}+\frac{1}{3})\right|}_{\text{可以看出绝对值小于1}}$.

                       如果$m-n$为偶数,则两两凑成一对即知式子绝对值小于第一项绝对值.综上只需取$N$使得$\frac{1}{n+1}<\varepsilon$即可.
                   \end{solution}
             \item $c_n=\frac{\sin2x}{2(2+\sin2x)}+\frac{\sin3x}{3(3+\sin3x)}+\cdots+\frac{\sin nx}{n(n+\sin nx)},n\in \mathbb{N}_+$.
                   \begin{solution}
                       显然有$\left|x_m-x_n\right|\leqslant\left|\frac{1}{(n+1)(n)}+\cdots+\frac{1}{m(m-1)}\right|$,裂项相消即可.
                   \end{solution}
         \end{enumerate}
     \end{exercise}

     \begin{exercise}
         设$a_n=\sin1+\frac{\sin2}{2!}+\cdots+\frac{\sin n}{n!},n\in \mathbb{N}_+$,证明:
         \begin{enumerate}
             \item 数列$\{a_n\}$有界,但不单调;
                   \begin{solution}
                       显然$a_n$不单调;而$a_n\leqslant\left|\sin1\right|+\cdots+\left|\frac{\sin n}{n!}\right|\leqslant\frac{1}{1!}+\cdots+\frac{1}{n!}$,由$\sum_{i=1}^\infty\frac{1}{i!}$收敛知$\{a_n\}$有界.
                   \end{solution}
             \item $\{a_n\}$收敛.
                   \begin{solution}
                       任意两项之差绝对值可以按照(1)放缩,由$\sum_{i=1}^\infty\frac{1}{i!}$收敛知使差绝对值$<\varepsilon$的$N$是存在的,按照Cauthy收敛准则知$\{a_n\}$收敛.
                   \end{solution}
         \end{enumerate}
     \end{exercise}

     \begin{exercise}
         设从某个数列$\{x_n\}$定义$x_n=\sum_{k=1}^na_k,y_n=\sum_{k=1}^n\left|a_k\right|,n\in \mathbb{N}_+$,若数列$\{y_n\}$收敛,证明数列$\{x_n\}$也收敛.
         \begin{note}
             本题可以看成是上一题和例题3.4.1的推广.
         \end{note}
     \end{exercise}
     \begin{solution}
         有$\left|x_m-x_n\right|\leqslant y_m-y_n$,而因为$\{y_n\}$收敛,由Cauthy收敛准则知对任意$\varepsilon$可以找到$N$使$y_m-y_n<\varepsilon$,而根据前边不等式知这个$N$对$\{x_n\}$也成立,即$\{x_n\}$也是基本数列,故$\{x_n\}$收敛.
     \end{solution}

     \begin{exercise}
         设$S_n=1+\frac{1}{2^p}+\frac{1}{3^p}+\cdots+\frac{1}{n^p},n\in \mathbb{N}_+$,其中$p\leqslant 1$,证明$\{S_n\}$发散.
     \end{exercise}
     \begin{solution}
         只需证明无论多大的$N$都可以找到$n,m>N$且$\left|x_m-x_n\right|>\varepsilon$(\hyperlink{2.4.p.2}{参照}),按照前边的做法,只需取n使得$n>N,n=2^k,m=2^{k+1}$,则$\left|x_m-x_n\right|>\frac{1}{2}$,由Cauthy收敛准则的反面表述知$S_n$发散.
     \end{solution}

     \begin{exercise}
         天文学中的Kepler(开普勒)方程$x-q\sin x=a(0<q<1)$是一个超越方程,没有求根公式.求近似解的一个方法是通过迭代.取定$x_1$,然后用递推公式$x_{n+1}=q\sin x_n+a,n\in \mathbb{N}_+$.证明这个方法的正确性.
     \end{exercise}
     \begin{solution}
         容易看出$f(x)=q\sin x+a$是一个$[a-q,a+q]$上的压缩映射,$q$就是一个压缩因子.而且由于$\sin x$的特性,从第二项开始就$\{x_n\}$就落入了$[a-q,a+q]$中,根据压缩映射原理知此函数的不动点,也即开普勒方程的解即为$\{x_n\}$的极限.
     \end{solution}
