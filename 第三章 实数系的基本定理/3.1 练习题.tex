\documentclass{exam}
\usepackage{amsmath,amssymb,ctex}

%uncomment \printanswers to show answers
\printanswers
%\unframedsolutions
\everymath{\displaystyle}

\begin{document}

\begin{center}
    %Title
    \Large 3.1 \, 确界的概念和确界存在定理
\end{center}

\begin{questions}
    %Questions goes here.
    \question
    试证明确界的唯一性.
    \begin{solution}
        反证.不妨设有$a>b$都为$A$上确界,

        取$0<\varepsilon<a-b$,则由确界定义知$\exists\, a_n>a-\varepsilon>b$且
        $a_n<a$;这与$b$也为上确界矛盾,故上确界唯一.下确界同理可证.
    \end{solution}

    \question
    设对每个$x\in A$成立$x<a$.问:在$\sup A<a$和$\sup A\leqslant a$中哪个是对的?
    \begin{solution}
        后者.显然$\sup A\leqslant a$必然成立.
        
        设$A=\left\{1-(\frac{1}{2})^n\,\bigg\lvert\,n\in N_+\right\}$,
        显然$x<1$对$x\in A$都成立,而无论多小的$\varepsilon$,都$\exists
        \,n\in N_+\,s.t.\,1-(\frac{1}{2})^n>1-\varepsilon$,即$\sup A=1$.
    \end{solution}
    \question
    设数集$A$以$\beta$为上界,又有数列$\{x_n\}\subset A$和$\lim_{n\to\infty}x_n=\beta.$证明$\beta=\sup A$.
    \begin{solution}
        由收敛定义可知无论多小的$\varepsilon>0$都$\exists\,N\in N_+
        \,s.t.\,n>N,\beta-x_n<\varepsilon\Rightarrow\beta-\varepsilon<x_n$.
        这就是上确界定义.
    \end{solution}
    \question
    求下列数集的上确界和下确界:
    \begin{parts}
        \part
        $\{x\in Q\,\lvert\, x>0\};$
        \begin{solution}
            
        \end{solution}
        \part
        $\{y\,\lvert\, y=x^2,x\in(-\frac{1}{2},1)\};$

        \part
        $\left\{\left(1+\frac{1}{n}\right)^n \,\bigg\lvert\, n\in N_+\right\};$

        \part
        $\{ne^{-n}\,\lvert\, n\in N_+\};$

        \part
        $\{\arctan x \,\lvert\, x\in (-\infty,+\infty)\};$

        \part
        $\left\{(-1)^n+\frac{1}{n}(-1)^{n+1}\,\biggl\lvert\, n\in N_+\right\};$

        \part
        $\left\{1+n\sin\frac{n\pi}{2}\,\bigg\lvert\, n\in N_+\right\}$.
    \end{parts}

    \question
    证明:
    \begin{parts}
        \part
        $\sup\{x_n+y_n\}\leqslant \sup\{x_n\}+\sup\{y_n\}$;

        \part
        $\inf\{x_n+y_n\}\geqslant\inf\{x_n\}+\inf\{y_n\}$.
    \end{parts}

    \question
    设有两个数集$A$和$B$,且对数集$A$中的任何一个数$x$和数集$B$中的任何一个数$y$成立不等式$x\leqslant y$.
    证明:$\sup\{x_n\}\leqslant\{y_n\}$.

    \question
    设数集$A$有上界,数集$B=\{x+c\,\lvert\, x\in A\}$,其中$c$是一个常数.证明:
    \[\sup B=\sup A+c,\inf B=\inf A+c.
        \]

    \question
    设$A,B$是两个有上界的数集,又有数集$C\subset \{x+y\,\lvert\, x\in A,y\in B\}$,则
    $\sup C\leqslant \sup A+\sup B$.举出成立严格不等号的例子.

    \question
    设$A,B$是两个有上界的数集,又有数集$C\supset  \{x+y\,\lvert\, x\in A,y\in B\}$,则
    $\sup C\geqslant \sup A+\sup B$.举出成立严格不等号的例子.
    \begin{solution}
        

        (合并以上两题可见:当且仅当$C=\{x+y\,\lvert\, x\in A,y\in B\}$时成立$\sup C=\sup A+\sup B$.)
    \end{solution}

\end{questions}

\end{document}

