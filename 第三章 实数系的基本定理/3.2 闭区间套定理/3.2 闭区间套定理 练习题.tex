\documentclass[cn,chinese,fontset]{elegantbook}
\usepackage{wrapfig}
\everymath{\displaystyle}

\begin{document}
    \chapter{实数系的基本定理}
        \section{闭区间套定理}
            \subsection{练习题}
            \begin{exercise}
                如果数列是$\{(-1)^n\}$,开始的区间是$[-1,1]$.试用例题3.2.2中的方法具体找出一个闭区间套和相应的收敛子列.又问: 你能否用这样的方法在这个例子中找出3个收敛子列?
            \end{exercise}
            \begin{solution}
                顺着例题操作来即可.

                取中点将$[-1,1]$分成两部分,两边都具有无穷多项,任取一边,不妨设取了$[0,1]$,则之后只能取$[\frac{1}{2},1],[\frac{3}{4},1],\cdots$.

                不能用这个方法获得3个收敛子列.在第一次选择之后就没得选择了(按这个方法的流程来的话).
            \end{solution}

            \begin{exercise}
                如闭区间套定理中的闭区间套改为开区间套$\{(a_n,b_n)\}$,其他条件不变,则可以举出例子说明结论不成立.
            \end{exercise}
            \begin{solution}
                如闭区间套$\left\{(0,(\frac{1}{2})^n)\right\}$,显然$0\notin\left\{(0,(\frac{1}{2})^n)\right\}$,而对$(0,\frac{1}{2})$之间的任意数,都存在N使$n>N$时该数不在之后的区间套中.

                也就是说$\textstyle\bigcap_{n=1}^\infty\left\{(0,(\frac{1}{2})^n)\right\}=\varnothing $
            \end{solution}

            \begin{exercise}
                如$\{(a_n,b_n)\}$为开区间套,数列$\{a_n\}$严格单调递增,数列$\{b_n\}$严格单调减少,
                又满足条件$a_n<b_n,n\in \mathsf{N}_+$,证明$\textstyle\bigcap_{n=1}^\infty(a_n,b_n)\neq\varnothing$.
            \end{exercise}
            \begin{solution}
                显然$\{a_n\},\{b_n\}$都是单调有界数列,不妨设分别收敛到$a,b$.

                如果$a\neq b$,则取$\xi \in (a,b)$即可;否则取$\xi=a=b$,由题知对任意n始终成立$a_n<\xi<b_n$,也即$\xi\in\textstyle\bigcap_{n=1}^\infty(a_n,b_n)$.

                故结论成立.
            \end{solution}

            \begin{exercise}
                用闭区间套定理证明确界存在定理.
            \end{exercise}
            \begin{solution}
                参照Bolzano二分法,$[a_1,b_1]$中$a_1$从$A$中选出,$b_1$为任意一个上界.如果相等则结束,$a_1=b_1$即为上确界.

                如果不等则取中点,若$A$中存在更大的数则令$a_2=\frac{a_1+b_1}{2}$,否则令$b_2=\frac{a_1+b_1}{2}$.按此流程下去知$[a_n,b_n]$满足
                
                \circled{1}区间套长度趋0
                
                \circled{2}$\textstyle\bigcap_{n=1}^\infty[a_n,b_n]$中元素满足$\geqslant A$中所有元素且$\leqslant $所有上界.
                
                由闭区间套定理知最后得到的单点集中元素即为所求上确界.下确界做法类似.
            \end{solution}
            
            \begin{exercise}
                用闭区间套定理证明单调有界数列的收敛定理.
            \end{exercise}
            \begin{solution}
                如果数列单调递增有上界,则类似上题,从$\{a_n\}$中任取一个作为初始区间套下界,任取一个上界作为初始区间套上界,再遵照Bolzano二分法即可得到结论.单调递减类似.
            \end{solution}

\end{document}