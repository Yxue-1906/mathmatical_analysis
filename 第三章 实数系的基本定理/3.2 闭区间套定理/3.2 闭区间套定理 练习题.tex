\documentclass[windows,list,answers]{BHCexam}
\usepackage{fontspec,wrapfig,xunicode-addon}
\setmainfont{Yu Mincho}
\setCJKsansfont{STSong}
\begin{document}
\everymath{\displaystyle}
\title{3.2 闭区间套定理\quad 练习题}

\maketitle

\begin{questions}
    \question
    如果数列是$\{(-1)^n\}$,开始的区间是$[-1,1]$.试用例题3.2.2中的方法具体找出
    一个闭区间套和相应的收敛子列.又问: 你能否用这样的方法在这个例子中找出3个收敛子列?
    \begin{solution}{1cm}
        \methodonly
        顺着例题操作来即可.

        取中点将$[-1,1]$分成两部分,两边都具有无穷多项,任取一边,不妨设取了$[0,1]$,
        则之后只能取$[\frac{1}{2},1],[\frac{3}{4},1],\cdots$.

        不能用这个方法获得3个收敛子列.在第一次选择之后就没得选择了( 按这个方法的流程
        来的话).
    \end{solution}

    \question
    如闭区间套定理中的闭区间套改为开区间套$\{(a_n,b_n)\}$,其他条件不变,则可以举出例子
    说明结论不成立.
    \begin{solution}{1cm}
        \methodonly
        如闭区间套$\left\{(0,(\frac{1}{2})^n)\right\}$,显然$0\notin 
        \left\{(0,(\frac{1}{2})^n)\right\}$,而对$(0,\frac{1}{2})$之间的任意数,都
        存在N使$n>N$时该数不在之后的区间套中.

        也就是说$\textstyle\bigcap_{n=1}^\infty\left\{(0,(\frac{1}{2})^n)\right\}=
        \emptyset $
    \end{solution}

    \question
    如$\{(a_n,b_n)\}$为开区间套,数列$\{a_n\}$严格单调递增,数列$\{b_n\}$严格单调减少,
    又满足条件$a_n<b_n,n\in \mathsf{N}_+$,证明
    $\textstyle\bigcap_{n=1}^\infty(a_n,b_n)\neq\emptyset$.
    \begin{solution}{1cm}
        \methodonly
        显然$\{a_n\},\{b_n\}$都是单调有界数列,不妨设分别收敛到$a,b$.

        如果$a\neq b$,则取$\xi \in (a,b)$即可;否则取$\xi=a=b$,由题知对任意n始终成立
        $a_n<\xi<b_n$,也即$\xi\in\textstyle\bigcap_{n=1}^\infty(a_n,b_n)$.

        故结论成立.
    \end{solution}

    \question
    用闭区间套定理证明确界存在定理.
    \begin{solution}{1cm}
        \methodonly
        参照Bolzano二分法,$[a_1,b_1]$中$a_1$从$A$中选出,$b_1$为任意一个上界.
        如果相等则结束,$a_1=b_1$即为上确界.

        如果不等则取中点,若$A$中存在更大的数则令$a_2=\frac{a_1+b_1}{2}$,否则
        令$b_2=\frac{a_1+b_1}{2}$.按此流程下去知$[a_n,b_n]$满足
        
        \textcircled{1}区间套长度趋0
        
        \textcircled{2}$\textstyle\bigcap_{n=1}^\infty[a_n,b_n]$中元素满足$\geqslant A$
        中所有元素且$\leqslant $所有上界.
        
        由闭区间套定理知最后得到的单点集中元素即为所求上确界.下确界做法类似.
    \end{solution}
    
    \question
    用闭区间套定理证明单调有界数列的收敛定理.
    \begin{solution}{1cm}
        \methodonly
        如果数列单调递增有上界,则类似上题,从$\{a_n\}$中任取一个作为初始区间套下界,
        任取一个上界作为初始区间套上界,再遵照Bolzano二分法即可得到结论.单调递减类似.
    \end{solution}

\end{questions}

\end{document}