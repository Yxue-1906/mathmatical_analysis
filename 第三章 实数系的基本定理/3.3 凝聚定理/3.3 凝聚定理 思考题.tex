% 使用 BHCexam 文档类,并传递选项
\documentclass[windows,list,answers]{BHCexam}
\usepackage{fontspec,wrapfig,xunicode-addon}
\setmainfont{Yu Mincho}
\setCJKsansfont{STSong}
\begin{document}
\everymath{\displaystyle}
\title{3.2 凝聚定理\quad 思考题}

\maketitle

注:凝聚定理(Bolzano-Weierstrass(布尔查诺-魏尔斯特拉斯)定理):有界数列必有收敛子列.

\begin{questions}
    \question
    凝聚定理在实数系中成立,在有理数集$\mathbb{Q} $中不成立.
    \begin{solution}{1cm}
        \methodonly
        这个在$\mathbb{Q}$中不成立指的是无法收敛到$\mathbb{Q}$中的某个数.
        也就是说凝聚定理的基础数列的收敛性不受保证.
        
        如$\left\{a_n=\sqrt{2}\text{小数点前n位处截断}\right\}$,即
        $a_1=1.4,a_2=1.41,a_3=1.414\cdots$

        这个数列在$\mathbb{Q}$中不收敛.而选出的子列的下标必须单调递增,
        也就是说无论以何种方式选出的子列在有理数集中都不收敛.

        可以猜出,凝聚定理不成立的一个充分条件是原数列在实数系中收敛到
        $\xi\in\mathbb{R}$且$\xi\notin\mathbb{Q}$,这样所有的子列在
        实数系中都收敛到$\xi$,这时在有理数集中凝聚定理就不成立了.

        结合实数系中的凝聚定理,可以改写一下表述,即有理数集上的有界数列
        不一定能选出收敛到有理数的子列,也即选出的所有收敛子列都收敛到实数.

        另一个例子就是
        \[a_n=\begin{cases}
            \sqrt{2}\text{小数点前n位处截断},\, 2\mid  n\\
            \sqrt{3}\text{小数点前n位处截断},\, 2\nmid n
        \end{cases}\]
        即$a_1=1.4,a_2=1.73,a_3=1.414\cdots$

        这个数列可以选出许多个在实数系中分别收敛到$\sqrt{2}$或$\sqrt{3}$,
        但这两个数在有理数集中都不存在.
    \end{solution}
    
\end{questions}

\end{document}