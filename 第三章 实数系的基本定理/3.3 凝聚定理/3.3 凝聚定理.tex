\documentclass[cn]{elegantbook}
\usepackage{wrapfig}

\begin{document}
    \chapter{实数系的基本定理}
    \section{凝聚定理}
    \subsection{思考题}
        \begin{example}
            凝聚定理在实数系中成立,在有理数集$\mathbb{Q} $中不成立.
        \end{example}
        \begin{solution}
            这个在$\mathbb{Q}$中不成立指的是无法收敛到$\mathbb{Q}$中的某个数.
            也就是说凝聚定理的基础数列的收敛性不受保证.
            
            如$a_n=\sqrt{2}$小数点前n位处截断,即
            $a_1=1.4,a_2=1.41,a_3=1.414\cdots$

            这个数列在$\mathbb{Q}$中不收敛.而选出的子列的下标必须单调递增,
            也就是说无论以何种方式选出的子列在有理数集中都不收敛.

            可以猜出,凝聚定理不成立的一个充分条件是原数列在实数系中收敛到
            $\xi\in\mathbb{R}$且$\xi\notin\mathbb{Q}$,这样所有的子列在
            实数系中都收敛到$\xi$,这时在有理数集中凝聚定理就不成立了.

            结合实数系中的凝聚定理,可以改写一下表述,即有理数集上的有界数列
            不一定能选出收敛到有理数的子列,也即选出的所有收敛子列都收敛到实数.

            另一个例子就是
            \[a_n=\begin{cases}
                \sqrt{2}\text{小数点前n位处截断},\, 2\mid  n\\
                \sqrt{3}\text{小数点前n位处截断},\, 2\nmid n
            \end{cases}\]
            即$a_1=1.4,a_2=1.73,a_3=1.414\cdots$

            这个数列可以选出许多个在实数系中分别收敛到$\sqrt{2}$或$\sqrt{3}$,
            但这两个数在有理数集中都不存在.
        \end{solution}
    \subsection{练习题}
        \begin{exercise}
            对于给定的数列$\{x_n\}$和数$a$,证明:在$a$的每个邻域中都有$\{x_n\}$
        的无穷多项的充分必要条件是,$a$是数列$\{x_n\}$某个子列的极限.
        \end{exercise}
        \begin{solution}
            充分性比较显然.由极限定义知,只需将$\varepsilon$取为邻域长度
            一半即可.下证必要性.
            
            可以选择一收敛于0的数列$\{\varepsilon_n\}$,
            由题知对于每个$\varepsilon_n\in\{\varepsilon_n\}$,都可以找出
            $a_n\in\{x_n\}$且在$a$的邻域内,这样就选出了一个收敛到$a$的子列.
        \end{solution}

        \begin{exercise}
            证明:有界数列发散的充分必要条件是存在两个收敛于不同极限值的子列.
        \end{exercise}
        \begin{solution}
            充分性显然.这也是之前证明的数列收敛的必要条件.下证必要性.

            利用Bolzano二分法,不过稍微做些改造.每次将选择的区间分为三个子区间,
            对于每个具有数列中无穷多项的子区间都进一步分割,
            如果某次分割时两端的区间都有数列中的无穷多项,那么分别在这两个区间中应用
            Bolzano二分法即可得到两个趋向不同极限的子列.

            下证在有限步中可以得到不相邻的两个或以上的子区间.
            反证,如果在进行的时候只能得到一个或者相邻的两个子区间,
            对于前者可知形成了一个闭区间套,应用闭区间套定理即知数列收敛.矛盾.

            对于后者,取左子区间的左端和右子区间的右端即可得到一列闭区间套
            同样可以得到数列收敛.矛盾.

            综上,定理成立.
        \end{solution}

        \begin{exercise}
            证明:若$\{x_n\}$无界,但不是无穷大量,则存在两个子列,其中一个
            子列收敛,另一个子列是无穷大量.
        \end{exercise}
        \begin{solution}
            由题知,数列无界,则任意选定一$M$,取任意一
            $\lvert x_{n_1}\rvert>M$作为
            子列第一项,然后选取
            $\lvert x_{n_2}\rvert>\lvert x_{n_1}\rvert$且$n_2>n_1$
            作为子列第二项,重复这个过程即可得到一无穷大量.

            由数列并非是无穷大量知,$\exists M>0,\forall N\in \mathbb{N}_+,
            \exists n>N,\lvert x_n\rvert<M$,这样可以选出一个有界子列,
            由凝聚定理知可以选出一个收敛于有限极限的子列.
        \end{solution}

        \begin{exercise}
            用凝聚定理证明单调有界数列的收敛定理.
        \end{exercise}
        \begin{solution}
            由凝聚定理知,单调有界数列存在一收敛子列,设收敛到$a$,
            且不妨设单调递增.

            对于任意小的$\varepsilon>0$,由于子列收敛,可以得到$N_k$
            使得$n_k>N_k$时$\lvert x_{n_k}-a\rvert<\varepsilon$.
            由于数列$\{x_n\}$单调,则$n>N_k$的任意$x_n$,都有
            $x_{n_k}<x_n<x_{n_{k+1}}$,也即也成立$\lvert x_n-a\rvert<\varepsilon$.
            也即$\{x_n\}$收敛到$a$.

            证毕.
        \end{solution}
\end{document}