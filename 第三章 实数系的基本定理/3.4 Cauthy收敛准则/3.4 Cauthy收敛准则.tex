\documentclass[cn,chinese,fontset]{elegantbook}
\usepackage{wrapfig}
\everymath{\displaystyle}

\begin{document}
    \chapter{实数系的基本定理}
        \section{Cauthy收敛准则}
        
        \begin{definition}{基本数列}{Cauchy Sequence}
            称数列$\{x_n\}$为\textbf{基本数列}(或\textbf{Cauthy数列}),如果对每个$\varepsilon>0$,存在$N$,使得对每一对正整数$n,m>N$,成立估计式$\lvert a_n-a_m\rvert<\varepsilon$.
        \end{definition}
        
        \begin{theorem}{Cauthy收敛准则}{Cauthy Theorem} 
            数列收敛的充分必要条件是该数列为基本数列.
        \end{theorem}
        
        \begin{definition}{压缩映射}{Contraction Mapping}
            设函数$f$在区间$[a,b]$上定义,$f([a,b])\subset [a,b]$,并存在一个常数$k$,满足$0<k<1$,使得对一切 $x,y\in [a,b]$成立不等式$\lvert f(x)-f(y)\rvert\leqslant k\lvert x-y\rvert$,则称$f$是$[a,b]$上的一个\textbf{压缩映射},称常数$k$为\textbf{压缩常数}.
        \end{definition}
        \begin{theorem}{压缩映射原理}{Contraction Mapping Theorem}
            设$f$是$[a,b]$上的一个压缩映射,则
            \begin{enumerate}
                \item $f$在$[a,b]$中存在唯一的不动点$\xi=f(\xi)$;
                \item 由任何初始值$a_0\in[a,b]$和递推公式$a_{n+1}=f(a_n),n\in N_+$生成的数列$\{a_n\}$一定收敛于$\xi$;
                \item 成立估计式$\lvert a_n-\xi\rvert\leqslant\frac{k}{1-k}\lvert a_n-a{n-1}\rvert$和$\lvert a_n-\xi\rvert\leqslant\frac{k^n}{1-k}\lvert a_1-a_0\rvert$(即事后估计与先验估计).
            \end{enumerate}
        \end{theorem}
            \subsection{思考题}
            \begin{example}
                Cauthy收敛准则在有理数集$\mathbb{Q}$中不不成立.
            \end{example}
            \begin{solution}
                原因和上一节相同,都是可能存在在实数系中的极限但是不在有理数集中.

                这个\href{https://www.zhihu.com/question/50995932/answer/866173110}{链接}有更加专业的解释,可以参考下图.

                \includegraphics[width=\linewidth]{Cauthy收敛准则在Q中不成立.png}
            \end{solution}
        
            \subsection{练习题}
            \begin{exercise}
                满足以下条件的数列$\{x_n\}$是否一定是基本数列?若回答"是",请做出证明;若回答"不一定是",请举出反例:
                \begin{enumerate}
                    \item 对每个$\varepsilon>0$,存在$N$,当$n>N$是,成立
                    $\lvert x_n-x_N\rvert<\varepsilon$;
                    \begin{solution}
                        是.对于任意$\varepsilon>0$,由题设知可以得到$N$,使得当$n>N$时成立$\lvert x_n-x_N\rvert<\frac{\varepsilon}{2},\lvert x_m-x_N\rvert<\frac{\varepsilon}{2}$,则$\lvert x_n-x_m\rvert<\lvert x_n-x_N\rvert+\lvert x_m-x_N\rvert<\varepsilon$.
                    \end{solution}
                    \item 对所有$n,p\in N_+$,成立不等式$\lvert x_{n+p}-x_n\rvert\leqslant\frac{p}{n}$;
                    \begin{solution}
                        不一定是.显然基本数列满足这个条件,但例如$a_n=\sum_{i=1}^{n}\frac{1}{i}$,显然成立题设条件,甚至我们可以直接得到估计,但是我们知道$\lim_{n\to\infty}a_n=+\infty$,所以这不一定是基本数列.
                    \end{solution}
                    \item 对所有$n,p\in N_+$,成立不等式$\lvert x_{n+p}-x_n\rvert\leqslant\frac{p}{n^2}$;
                    \begin{solution}
                        是.根据题设,我们可以得到更精确的估计:$\lvert x_{n+p}-x_n\rvert<\lvert x_{n+p}-x_{n+p-1}\rvert+\cdots+\lvert x_{n+1}-x_{n}\rvert\leqslant\sum_{i=n}^{n+p-1}\frac{1}{i^2}$.根据我们之前得到的结论,$S_n=\sum_{i=1}^{\infty}\frac{1}{i^2}$收敛,所以只需$N$足够大就可以得到$\lvert x_{n+p}-x_n\rvert<S_{n+p}-S_n<\varepsilon$.
                    \end{solution}
                    \item 对每个正整数$p$,成立$\lim_{n\to\infty}(x_n-x_{n+p})=0$.
                    \begin{solution}
                        不一定是.
                    \end{solution}
                \end{enumerate}
            \end{exercise}
\end{document}