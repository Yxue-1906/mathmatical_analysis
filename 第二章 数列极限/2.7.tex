\documentclass{exam}
\usepackage{amsmath,amssymb,ctex}

%解除注释\printanswers可以显示答案,解除注释\unframedsolution消去答案带有的框
\printanswers
%\unframedsolutions
\everymath{\displaystyle}

\begin{document}

\begin{center}
    \Large 第二章\, 数列极限\, 第一组参考题
\end{center}

\begin{questions}

    \question
    设$\{a_{2k-1}\},\{a_{2k}\},\{a_{3k}\}$都收敛,证明:$\{a_n\}$收敛.
    \begin{solution}
        设$\{a_{2k-1}\}$,收敛到$p$,$\{a_{2k}\}$收敛到$q$,而$\{a_{3k}\}$中可以选出全属于$\{a_{2k-1}\}$或$\{a_{2k}\}$的子列(事实上$\{a_{3k}\}$中的项交替地从$\{a_{2k-1}\}$和$\{a_{2k}\}$中取出),
        由于收敛数列子列收敛于同一极限,也即$p=q$,而前证得$\{a_{2k-1}\},\{a_{2k}\}$收敛于同一极限则$\{a_n\}$也收敛于同一极限知$\{a_n\}$收敛.
    \end{solution}

    \question
    设$\{a_n\}$有界,且满足条件$a_n\leqslant a_{n+2},a_n\leqslant a_{n+3},n\in N_+$,证明:$\{a_n\}$收敛.
    \begin{solution}
        由题知$\{a_{2k-1}\},\{a_{2k}\},\{a_{3k}\}$收敛,由上一小题知命题成立.
    \end{solution}

    \question
    设$\{a_n+a_{n+1}\}$和$\{a_n+a_{n+2}\}$都收敛,证明:$\{a_n\}$收敛.
    \begin{solution}
        由题知$\{a_{n+2}-a_{n+1}\}$也收敛,而实质上这与$\{a_{n+1}-a_n\}$没有区别,与$\{a_{n+1}+a_n\}$相减即得$\{a_n\}$收敛.
    \end{solution}

    \question
    设数列$\{a_n\}$收敛于0,又存在极限$\lim _{n\to\infty} \left| \frac {a_{n+1}}{a_n} \right|=a$.证明:$a\leqslant 1$.
    \begin{solution}
        反证.

        首先$a_n\neq0$,否则$\frac{a_{n+1}}{a_n}$某项没有意义.

        若$a>1$,则$\exists a'$使得$1<a'<a$,而取$\varepsilon<a-a'$,无论多大的$N$,总
        $\exists n>N,a_n\geqslant a_N\cdot a^{n-N}>\varepsilon$,与$\{a_n\}$收敛到0矛盾.
    \end{solution}

    \question
    设$a_n=\sum\limits _{k=1}^n \left(\sqrt{1+\frac{k}{n^2}}-1\right)$,$n\in N_+$,计算$\lim\limits _{n\to \infty}a_n$.
    \begin{solution}
        放缩.

        $a_n=\sum_{k=1}^n \frac{\frac{k}{n^2}}{\sqrt{1+\frac{k}{n^2}}+1}$

        而有$\frac{n+1}{2n}\cdot \frac{1}{\sqrt{1+\frac{1}{n^2}}+1}\leqslant a_n \leqslant \frac{n+1}{2n}\cdot\frac{1}{\sqrt{1+\frac{1}{n}}+1}$(放成同一分母)

        令$n$趋向$\infty$由夹逼定理即知$\lim_{n\to\infty}a_n=\frac{1}{2}$.

        说明:直接放缩会放过$\Longrightarrow n\cdot\left(\sqrt{1+\frac{1}{n^2}}-1\right)=
        \frac{n\cdot\frac{1}{n^2}}{\sqrt{1+\frac{1}{n^2}}+1}\rightarrow 0$
    \end{solution}

    \question
    用$p(n)$表示能整除$n$的素数的个数,证明:$\lim _{n\to \infty}\frac{p(n)}{n}=0$.
    \begin{solution}
        设$p_n$为第$n$个质数.可以这样估计一个用于夹逼的上数列:

        当$\prod_{i=1}^{k-1}p_i\leqslant n\leqslant\prod_{i=1}^kp_i$
        (也就是夹在$p_1p_2\cdots p_{k-1}$和$p_1p_2\cdots p_k$之间),

        则$\frac{p(n)}{n}\leqslant \frac{k}{p_1p_2\cdots p_k}<\frac{k}{k!}$,由夹逼定理知原极限为0.

        说明:事实上这就是一个对$p(n)$的一个估计,换个角度考虑,就是怎么使$\frac{p(n)}{n}$尽可能大,
        也就是变相估计上界吧.
    \end{solution}
    
    \question
    设$a_0,a_1,\cdots,a_p$是$p+1$个给定的数,且满足条件$a_0+a_1+\cdots+a_p=0$.求

    $\lim\limits _{n\to \infty}\left(a_0\sqrt{n}+a_1\sqrt{n+1}+\cdots+a_p\sqrt{n+p}\right)$.
    \begin{solution}
        
        $\left|a_0\sqrt{n}+\cdots+a_p\sqrt{n+p}-0\right|=$

        $\left|a_0(\sqrt{n}-\sqrt{n})+a_1(\sqrt{n+1}-\sqrt{n})+\cdots+a_p(\sqrt{n+p}-\sqrt{n})\right|\leqslant$

        $\left|a_0(\sqrt{n}-\sqrt{n})\right|+\left|a_1(\sqrt{n+1}-\sqrt{n})\right|+\cdots=$

        $\left|a_0\cdot0\right|+\left|a_1\cdot\frac{1}{\sqrt{n+1}+\sqrt{n}}\right|+\cdots+\left|\frac{a_p}{\sqrt{n+p}+\sqrt{n}}\right|\to 0$

        故极限为0.
    \end{solution}

    

    \question
    证明:当$0<k<1$时,$\lim _{n\to\infty} \left[(1+n)^k-n^k\right]=0$.
    \begin{solution}
        显然$0\leqslant(1+n)^k-n^k=n^k\left[(1+\frac{1}{n})^k-1\right]\leqslant 
        n^k\left[(1+\frac{1}{n})-1\right]=\frac{n^k}{n}\to0$.

        说明:另一思路也可以是证明单调有界,但是没想到怎么证明单调减...
    \end{solution}

    \question
    \begin{parts}
        \part
        设$\{a_n\}$收敛,令$y_n=n(x_n-x_{n-1}),n\in N_+$,问$\{y_n\}$是否收敛?
        \begin{solution}
            不一定.
            \[y_n=\begin{cases}
                x_{n-1}+\frac{1}{n}, \, n\text{为完全平方数} \\
                x_{n-1}, \, \text{其他}
            \end{cases}\]
            可知$\{x_n\}$不超过$\sum_{n = 1}^{\infty}\frac{1}{n^2} $,前证它收敛.而对于
            $\varepsilon<1$,无论多大的$N$总$\exists n>N$且$y_n=1$,即$\{y_n\}$发散.
        \end{solution}

        \part
        在上一小题中,若$\{y_n\}$也收敛,证明:$\{y_n\}$收敛于0.
        \begin{solution}
            设$\{y_n\}$收敛到$a$,由Stolz知$\left\{\frac{x_n-x_{n-1}}{\frac{1}{n}}\right\}$极限若存在则与
            $\frac{x_n}{\sum_{i=1}^\infty\frac{1}{i}}$相同,显然此数列收敛到0(分母$\to\infty$),而由题知
            $\left\{\frac{x_n-x_{n-1}}{\frac{1}{n}}\right\}$收敛,也即$a=0$.
        \end{solution}
    \end{parts}

    \question
    \begin{parts}
        \part
        设正数列$\{a_n\}$满足条件$\lim_{n\to\infty}\frac{a_n}{a_{n+1}}=0$,证明:$\{a_n\}$是正无穷大量.
        \begin{solution}
            事实上,当$n$足够大时$\left|\frac{a_n}{a_{n+1}}\right|=\frac{a_n}{a_{n+1}}<\varepsilon<1$.

            则当$n>N$时,$\frac{1}{a_n}<\frac{1}{a_N}\cdot\varepsilon^{n-N}$,即
            $a_n>\frac{a_N}{\varepsilon^{n-N}}>M$成立,($M$为一给定的任意大的数),也即$\{a_n\}$为无穷大量.
        \end{solution}

        \part
        设正数列$\{a_n\}$满足条件$\lim_{n\to\infty}\frac{a_n}{a_{n+1}+a_{n+2}}=0$,证明:$\{a_n\}$无界.
        \begin{solution}
            反证.假设$\{a_n\}$有上界$M$;而因为$\lim_{n\to\infty}\frac{a_n}{a_{n+1}+a_{n+2}}=0$
            $\Rightarrow $对$\varepsilon=\frac{1}{4},\exists N$使得当$n>N$时$\frac{a_n}{a_{n+1}+a_{n+2}}<\varepsilon=\frac{1}{4}$.
            那么可以推出$4a_N<a_{N+1}+a_{N+2},16a_N<a_{N+2}+2a_{N+3}+a_{N+4},\cdots$.

            因为$M$固定,故$\exists m$使得$2^ma_N>M$,即$2^mM<4^ma_N<\underbrace{\cdots}_{2^m\text{个}\{a_n\}\text{中的项}} $.

            由抽屉原理知,至少有一个项$>M$,与假设矛盾.故原命题成立.

            说明:只说无界是正确的,不一定是正无穷大量.$\{a_n\}$可以是$1!,1,2!,1,3!,1,\cdots$,这也满足题设.
        \end{solution}
    \end{parts}

    \question
    证明:$\left(\frac{n}{3}\right)^n<n!<\left(\frac{n}{2}\right)^n$,其中右边的不等式当$n\geqslant 6$时成立.
    \begin{solution}
        与2.5练习题7.证法完全类似.
    \end{solution}

    \question
    证明:$\left(\frac{n}{e}\right)^n<n!<e\left(\frac{n}{2}\right)^n$.
    \begin{solution}
        与2.5练习题7.证法完全类似.
    \end{solution}

    \question
    (对于命题2.5.4的改进)证明:
    \begin{parts}
        \part
        $n\geqslant2$时成立:$1+1+\frac{1}{2!}+\cdots+\frac{1}{n!}+\frac{1}{n!n}=
        3-\frac{1}{2!1\cdot2}-\cdots-\frac{1}{n!(n-1)n}$;
        \begin{solution}
            也即证$\frac{1}{2!}+\cdots+\frac{1}{n!}+\frac{1}{n!n}=1-\frac{1}{2!1\cdot2}-\cdots-\frac{1}{n!(n-1)n}$

            用数归,显然$n=2$时$\frac{1}{2!}+\frac{1}{2!2}=1-\frac{1}{2!1\cdot2}$;

            假设$n=k$时成立,即$\frac{1}{2!}+\cdots+\frac{1}{k!}+\frac{1}{k!k}=1-\frac{1}{2!1\cdot2}-\cdots-\frac{1}{k!(k-1)k}$,
            
            当$n=k+1$时,$\frac{1}{2!}+\cdots+\frac{1}{k!}+\frac{1}{(k+1)!}+\frac{1}{(k+1)!(k+1)}=
            1-\frac{1}{2!1\cdot2}-\cdots-\frac{1}{k!(k-1)k}+\frac{1}{(k+1)!}+\frac{1}{(k+1)!(k+1)}-\frac{1}{k!k}$.

            而通分后知$\frac{1}{(k+1)!}+\frac{1}{(k+1)!(k+1)}-\frac{1}{k!k}=-\frac{1}{(k+1)!k(k+1)}$,

            即$\frac{1}{2!}+\cdots+\frac{1}{k!}+\frac{1}{(k+1)!}+\frac{1}{(k+1)!(k+1)}=
            1-\frac{1}{2!1\cdot2}-\cdots-\frac{1}{k!(k-1)k}-\frac{1}{(k+1)!k(k+1)}$,也即命题对$n=k+1$也成立,
            由数归知对所有$n\geqslant2$都成立.
        \end{solution}

        \part
        $e=3-\lim_{n\to\infty}\sum_{k=0}^n\frac{1}{(k+2)!(k+1)(k+2)}$;
        \begin{solution}
            由(1)知$3-\lim_{n\to\infty}\sum_{k=0}^n\frac{1}{(k+2)!(k+1)(k+2)}=
            \lim_{n\to\infty}\left(\sum_{k=0}^n\frac{1}{k!}+\frac{1}{n!n}\right)$

            而前已证$\lim_{n\to\infty}\sum_{k=0}^n\frac{1}{k!}=e$,且显然$\lim_{n\to\infty}\frac{1}{n!n}=0$,
            由极限的四则运算知等式成立.
        \end{solution}

        \part
        用$\sum_{k=0}^n\frac{1}{k!}+\frac{1}{n!n}$计算$e$要比不加上最后一项好得多.
        \begin{solution}
            由(1)的右边可以看出$\lim_{n\to\infty}\left(\sum_{k=0}^n\frac{1}{k!}+\frac{1}{n!n}\right)$是大于$e$的.

            也就是说,$\lim_{n\to\infty}\sum_{k=0}^n\frac{1}{k!}$单调增从下方逼近$e$,
            $\lim_{n\to\infty}\left(\sum_{k=0}^n\frac{1}{k!}+\frac{1}{n!n}\right)$单调减从上方逼近$e$.

            而他们和$e$的误差为$\alpha_n=e-\sum_{k=0}^n\frac{1}{k!}=\sum_{k=n+1}^\infty\frac{1}{k!},
            \beta_n=\sum_{k=0}^n\frac{1}{k!}+\frac{1}{n!n}-e=\sum_{k=n+1}^\infty\frac{1}{k!(k-1)k}$.

            显然$\beta_n<\alpha_n$也就是说后者收敛得更快.
        \end{solution}

    \end{parts}

    \question
    设$a_n=1+\frac{1}{\sqrt{2}}+\frac{1}{\sqrt{3}}+\cdots+\frac{1}{\sqrt{n}}-2\sqrt{n},n\in N_+$,
    证明:$\{a_n\}$收敛.
    \begin{solution}
        $\because\frac{1}{\sqrt{k}}=\frac{2}{2\sqrt{k}}<\frac{2}{\sqrt{k}+\sqrt{k-1}}=2\left(\sqrt{k}-\sqrt{k-1}\right)\Rightarrow 
        a_n<\left(-\sqrt{0}+\sqrt{1}\right)+\cdots+\left(-\sqrt{n-1}+\sqrt{n}\right)-2\sqrt{n}=0$.

        而考虑$a_{n+1}-a_n$,有差值为$\frac{1}{\sqrt{n+1}}+2(\sqrt{n+1}-\sqrt{n})>0$,故$a_n$单调递增有上界,即$\{a_n\}$收敛.
    \end{solution}

    \question
    设已知存在极限$\lim_{n\to\infty}\frac{a_1+a_2+\cdots+a_n}{n}$,证明:$\lim_{n\to\infty}\frac{a_n}{n}=0$.
    \begin{solution}
        (有点取巧?)

        $\lim_{n\to\infty}\frac{a_1+\cdots+a_n}{n}=
        \lim_{n\to\infty}\frac{a_1+\cdots+a_{n-1}}{n-1}\cdot\frac{n-1}{n}+\frac{a_n}{n}=
        \lim_{n\to\infty}\frac{a_1+\cdots+a_{n-1}}{n-1}\cdot\frac{n-1}{n}+\lim_{n\to\infty}\frac{a_n}{n}$

        而$\lim_{n\to\infty}\frac{a_1+\cdots+a_{n-1}}{n-1}\cdot\frac{n-1}{n}=\lim_{n\to\infty}\frac{a_1+\cdots+a_n}{n}$

        故$\lim_{n\to\infty}\frac{a_n}{n}=0$.
    \end{solution}

    \question
    证明:$\lim_{n\to\infty}(n!)^{1/n^2}=1$.
    \begin{solution}
        $\sqrt[n]{n!}<\sqrt[n]{n^n}=n$,而前已证$\lim_{n\to\infty}\sqrt[n]{n}=1$,又显然$(n!)^{1/n^2}>1$,
        由夹逼定理得命题成立.
    \end{solution}

    \question
    设对每个$n$有$x_n<1$和$(1-x_n)x_{n+1}\geqslant\frac{1}{4}$,证明$\{x_n\}$收敛,并求其极限.
    \begin{solution}
        (解法类似2.6练习题7.)
        
        先证$n\geqslant2$时$x_n$都为正数.

        因为$x_n<1$,所以$(1-x_n)>0$恒成立,故$x_{n+1}>0$,也就是说从$n=2$开始$\{x_n\}$都为正数.

        再证$\{x_n\}\leqslant\frac{1}{2}$,

        $(1-x_{n-1})x_n\geqslant\frac{1}{4}\Rightarrow x_n\geqslant\frac{1}{4(1-x_{n-1})}$,
        但$x_n<1$,所以$\frac{1}{4(1-x_{n-1})}<1\Rightarrow x_{n-1}<\frac{3}{4}$;同理
        $x_{n-1}<\frac{3}{4}\Rightarrow x_{n-2}<\frac{2}{3}\cdots$,若$x_n<m$,则$x_{n-1}<1-\frac{1}{4m}$,
        易证这是个单调减趋向$\frac{1}{2}$的序列,令n趋向无穷便得到$\{x_n\}$每一项都$\leqslant\frac{1}{2}$.

        从下方夹逼,
        
        $x_1<1\Rightarrow(1-x_1)>0\Rightarrow x_2>0\Rightarrow(1-x_2)<1\Rightarrow x_3>\frac{1}{4}
        \Rightarrow x_4>\frac{1}{3}\cdots$,同样可证这是一个单调增趋向$\frac{1}{2}$的序列,由夹逼定理知
        $\{x_n\}$收敛且极限为$\frac{1}{2}$.
    \end{solution}

    \question
    设$a_1=b,a_2=c$,再$n\geqslant3$时,$a_n=\frac{a_{n-1}+a_{n-2}}{2}$,证明$\{a_n\}$收敛,并求其极限.
    \begin{solution}
        一个方法是用特征根法解出通项为$a_n=(\frac{1}{3}b+\frac{2}{3}c)+\frac{4}{3}(c-b)(-\frac{1}{2})^n$,
        令n趋向无穷便得到极限为$\frac{1}{3}b+\frac{2}{3}c$.

        另外也可以像提示中那样证明奇数项和偶数项分别单调互为上下界.
    \end{solution}

    \question
    设$a,b,c$是三个给定的实数,令$a_1=a,b_1=b,c_1=c$,并以递推公式定义
    \[a_{n+1}=\frac{b_n+c_n}{2},b_{n+1}=\frac{a_n+c_n}{2},c_{n+1}=\frac{a_n+b_n}{2},n\in N+\].
    求这三个数列的极限.
    \begin{solution}
        注意到有$a_n+b_n+c_n=a+b+c$,且$a_{n+1}-b_{n+1}=-\frac{1}{2}(a_n-b_n)\Rrightarrow
        a_n-b_n=\left(-\frac{1}{2}\right)^n(a-b)$,同理对$b_n,c_n$也可以得到类似的式子,联立可以解出
        $\{a_n\},\{b_n\},\{c_n\}$,最后可以得到极限为$\frac{a+b+c}{3}$.
    \end{solution}

    \question
    \begin{parts}
        \part
        设$a_1>b_1>0,a_{n+1}=\frac{2a_nb_n}{a_n+b_n},b_{n+1}=\sqrt{a_{n+1}b_n},n\in N+$,证明:
        $\{a_n\}$和$\{b_n\}$收敛于同一极限.
        \begin{solution}
            
        \end{solution}

        \part
        在$a_1=2\sqrt{3},b_1=3$时,证明上述极限等于单位圆的半周长$\pi$.(这里可以利用极限
        $\lim_{n\to\infty}n\sin\frac{\pi}{n}=\pi.)$
        \begin{solution}
            
        \end{solution}
    \end{parts}

\end{questions}

\end{document}

